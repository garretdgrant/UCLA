
% Default to the notebook output style

    


% Inherit from the specified cell style.




    
\documentclass[11pt]{article}

    
    
    \usepackage[T1]{fontenc}
    % Nicer default font (+ math font) than Computer Modern for most use cases
    \usepackage{mathpazo}

    % Basic figure setup, for now with no caption control since it's done
    % automatically by Pandoc (which extracts ![](path) syntax from Markdown).
    \usepackage{graphicx}
    % We will generate all images so they have a width \maxwidth. This means
    % that they will get their normal width if they fit onto the page, but
    % are scaled down if they would overflow the margins.
    \makeatletter
    \def\maxwidth{\ifdim\Gin@nat@width>\linewidth\linewidth
    \else\Gin@nat@width\fi}
    \makeatother
    \let\Oldincludegraphics\includegraphics
    % Set max figure width to be 80% of text width, for now hardcoded.
    \renewcommand{\includegraphics}[1]{\Oldincludegraphics[width=.8\maxwidth]{#1}}
    % Ensure that by default, figures have no caption (until we provide a
    % proper Figure object with a Caption API and a way to capture that
    % in the conversion process - todo).
    \usepackage{caption}
    \DeclareCaptionLabelFormat{nolabel}{}
    \captionsetup{labelformat=nolabel}

    \usepackage{adjustbox} % Used to constrain images to a maximum size 
    \usepackage{xcolor} % Allow colors to be defined
    \usepackage{enumerate} % Needed for markdown enumerations to work
    \usepackage{geometry} % Used to adjust the document margins
    \usepackage{amsmath} % Equations
    \usepackage{amssymb} % Equations
    \usepackage{textcomp} % defines textquotesingle
    % Hack from http://tex.stackexchange.com/a/47451/13684:
    \AtBeginDocument{%
        \def\PYZsq{\textquotesingle}% Upright quotes in Pygmentized code
    }
    \usepackage{upquote} % Upright quotes for verbatim code
    \usepackage{eurosym} % defines \euro
    \usepackage[mathletters]{ucs} % Extended unicode (utf-8) support
    \usepackage[utf8x]{inputenc} % Allow utf-8 characters in the tex document
    \usepackage{fancyvrb} % verbatim replacement that allows latex
    \usepackage{grffile} % extends the file name processing of package graphics 
                         % to support a larger range 
    % The hyperref package gives us a pdf with properly built
    % internal navigation ('pdf bookmarks' for the table of contents,
    % internal cross-reference links, web links for URLs, etc.)
    \usepackage{hyperref}
    \usepackage{longtable} % longtable support required by pandoc >1.10
    \usepackage{booktabs}  % table support for pandoc > 1.12.2
    \usepackage[inline]{enumitem} % IRkernel/repr support (it uses the enumerate* environment)
    \usepackage[normalem]{ulem} % ulem is needed to support strikethroughs (\sout)
                                % normalem makes italics be italics, not underlines
    

    
    
    % Colors for the hyperref package
    \definecolor{urlcolor}{rgb}{0,.145,.698}
    \definecolor{linkcolor}{rgb}{.71,0.21,0.01}
    \definecolor{citecolor}{rgb}{.12,.54,.11}

    % ANSI colors
    \definecolor{ansi-black}{HTML}{3E424D}
    \definecolor{ansi-black-intense}{HTML}{282C36}
    \definecolor{ansi-red}{HTML}{E75C58}
    \definecolor{ansi-red-intense}{HTML}{B22B31}
    \definecolor{ansi-green}{HTML}{00A250}
    \definecolor{ansi-green-intense}{HTML}{007427}
    \definecolor{ansi-yellow}{HTML}{DDB62B}
    \definecolor{ansi-yellow-intense}{HTML}{B27D12}
    \definecolor{ansi-blue}{HTML}{208FFB}
    \definecolor{ansi-blue-intense}{HTML}{0065CA}
    \definecolor{ansi-magenta}{HTML}{D160C4}
    \definecolor{ansi-magenta-intense}{HTML}{A03196}
    \definecolor{ansi-cyan}{HTML}{60C6C8}
    \definecolor{ansi-cyan-intense}{HTML}{258F8F}
    \definecolor{ansi-white}{HTML}{C5C1B4}
    \definecolor{ansi-white-intense}{HTML}{A1A6B2}

    % commands and environments needed by pandoc snippets
    % extracted from the output of `pandoc -s`
    \providecommand{\tightlist}{%
      \setlength{\itemsep}{0pt}\setlength{\parskip}{0pt}}
    \DefineVerbatimEnvironment{Highlighting}{Verbatim}{commandchars=\\\{\}}
    % Add ',fontsize=\small' for more characters per line
    \newenvironment{Shaded}{}{}
    \newcommand{\KeywordTok}[1]{\textcolor[rgb]{0.00,0.44,0.13}{\textbf{{#1}}}}
    \newcommand{\DataTypeTok}[1]{\textcolor[rgb]{0.56,0.13,0.00}{{#1}}}
    \newcommand{\DecValTok}[1]{\textcolor[rgb]{0.25,0.63,0.44}{{#1}}}
    \newcommand{\BaseNTok}[1]{\textcolor[rgb]{0.25,0.63,0.44}{{#1}}}
    \newcommand{\FloatTok}[1]{\textcolor[rgb]{0.25,0.63,0.44}{{#1}}}
    \newcommand{\CharTok}[1]{\textcolor[rgb]{0.25,0.44,0.63}{{#1}}}
    \newcommand{\StringTok}[1]{\textcolor[rgb]{0.25,0.44,0.63}{{#1}}}
    \newcommand{\CommentTok}[1]{\textcolor[rgb]{0.38,0.63,0.69}{\textit{{#1}}}}
    \newcommand{\OtherTok}[1]{\textcolor[rgb]{0.00,0.44,0.13}{{#1}}}
    \newcommand{\AlertTok}[1]{\textcolor[rgb]{1.00,0.00,0.00}{\textbf{{#1}}}}
    \newcommand{\FunctionTok}[1]{\textcolor[rgb]{0.02,0.16,0.49}{{#1}}}
    \newcommand{\RegionMarkerTok}[1]{{#1}}
    \newcommand{\ErrorTok}[1]{\textcolor[rgb]{1.00,0.00,0.00}{\textbf{{#1}}}}
    \newcommand{\NormalTok}[1]{{#1}}
    
    % Additional commands for more recent versions of Pandoc
    \newcommand{\ConstantTok}[1]{\textcolor[rgb]{0.53,0.00,0.00}{{#1}}}
    \newcommand{\SpecialCharTok}[1]{\textcolor[rgb]{0.25,0.44,0.63}{{#1}}}
    \newcommand{\VerbatimStringTok}[1]{\textcolor[rgb]{0.25,0.44,0.63}{{#1}}}
    \newcommand{\SpecialStringTok}[1]{\textcolor[rgb]{0.73,0.40,0.53}{{#1}}}
    \newcommand{\ImportTok}[1]{{#1}}
    \newcommand{\DocumentationTok}[1]{\textcolor[rgb]{0.73,0.13,0.13}{\textit{{#1}}}}
    \newcommand{\AnnotationTok}[1]{\textcolor[rgb]{0.38,0.63,0.69}{\textbf{\textit{{#1}}}}}
    \newcommand{\CommentVarTok}[1]{\textcolor[rgb]{0.38,0.63,0.69}{\textbf{\textit{{#1}}}}}
    \newcommand{\VariableTok}[1]{\textcolor[rgb]{0.10,0.09,0.49}{{#1}}}
    \newcommand{\ControlFlowTok}[1]{\textcolor[rgb]{0.00,0.44,0.13}{\textbf{{#1}}}}
    \newcommand{\OperatorTok}[1]{\textcolor[rgb]{0.40,0.40,0.40}{{#1}}}
    \newcommand{\BuiltInTok}[1]{{#1}}
    \newcommand{\ExtensionTok}[1]{{#1}}
    \newcommand{\PreprocessorTok}[1]{\textcolor[rgb]{0.74,0.48,0.00}{{#1}}}
    \newcommand{\AttributeTok}[1]{\textcolor[rgb]{0.49,0.56,0.16}{{#1}}}
    \newcommand{\InformationTok}[1]{\textcolor[rgb]{0.38,0.63,0.69}{\textbf{\textit{{#1}}}}}
    \newcommand{\WarningTok}[1]{\textcolor[rgb]{0.38,0.63,0.69}{\textbf{\textit{{#1}}}}}
    
    
    % Define a nice break command that doesn't care if a line doesn't already
    % exist.
    \def\br{\hspace*{\fill} \\* }
    % Math Jax compatability definitions
    \def\gt{>}
    \def\lt{<}
    % Document parameters
    \title{HW1}
    
    
    

    % Pygments definitions
    
\makeatletter
\def\PY@reset{\let\PY@it=\relax \let\PY@bf=\relax%
    \let\PY@ul=\relax \let\PY@tc=\relax%
    \let\PY@bc=\relax \let\PY@ff=\relax}
\def\PY@tok#1{\csname PY@tok@#1\endcsname}
\def\PY@toks#1+{\ifx\relax#1\empty\else%
    \PY@tok{#1}\expandafter\PY@toks\fi}
\def\PY@do#1{\PY@bc{\PY@tc{\PY@ul{%
    \PY@it{\PY@bf{\PY@ff{#1}}}}}}}
\def\PY#1#2{\PY@reset\PY@toks#1+\relax+\PY@do{#2}}

\expandafter\def\csname PY@tok@w\endcsname{\def\PY@tc##1{\textcolor[rgb]{0.73,0.73,0.73}{##1}}}
\expandafter\def\csname PY@tok@c\endcsname{\let\PY@it=\textit\def\PY@tc##1{\textcolor[rgb]{0.25,0.50,0.50}{##1}}}
\expandafter\def\csname PY@tok@cp\endcsname{\def\PY@tc##1{\textcolor[rgb]{0.74,0.48,0.00}{##1}}}
\expandafter\def\csname PY@tok@k\endcsname{\let\PY@bf=\textbf\def\PY@tc##1{\textcolor[rgb]{0.00,0.50,0.00}{##1}}}
\expandafter\def\csname PY@tok@kp\endcsname{\def\PY@tc##1{\textcolor[rgb]{0.00,0.50,0.00}{##1}}}
\expandafter\def\csname PY@tok@kt\endcsname{\def\PY@tc##1{\textcolor[rgb]{0.69,0.00,0.25}{##1}}}
\expandafter\def\csname PY@tok@o\endcsname{\def\PY@tc##1{\textcolor[rgb]{0.40,0.40,0.40}{##1}}}
\expandafter\def\csname PY@tok@ow\endcsname{\let\PY@bf=\textbf\def\PY@tc##1{\textcolor[rgb]{0.67,0.13,1.00}{##1}}}
\expandafter\def\csname PY@tok@nb\endcsname{\def\PY@tc##1{\textcolor[rgb]{0.00,0.50,0.00}{##1}}}
\expandafter\def\csname PY@tok@nf\endcsname{\def\PY@tc##1{\textcolor[rgb]{0.00,0.00,1.00}{##1}}}
\expandafter\def\csname PY@tok@nc\endcsname{\let\PY@bf=\textbf\def\PY@tc##1{\textcolor[rgb]{0.00,0.00,1.00}{##1}}}
\expandafter\def\csname PY@tok@nn\endcsname{\let\PY@bf=\textbf\def\PY@tc##1{\textcolor[rgb]{0.00,0.00,1.00}{##1}}}
\expandafter\def\csname PY@tok@ne\endcsname{\let\PY@bf=\textbf\def\PY@tc##1{\textcolor[rgb]{0.82,0.25,0.23}{##1}}}
\expandafter\def\csname PY@tok@nv\endcsname{\def\PY@tc##1{\textcolor[rgb]{0.10,0.09,0.49}{##1}}}
\expandafter\def\csname PY@tok@no\endcsname{\def\PY@tc##1{\textcolor[rgb]{0.53,0.00,0.00}{##1}}}
\expandafter\def\csname PY@tok@nl\endcsname{\def\PY@tc##1{\textcolor[rgb]{0.63,0.63,0.00}{##1}}}
\expandafter\def\csname PY@tok@ni\endcsname{\let\PY@bf=\textbf\def\PY@tc##1{\textcolor[rgb]{0.60,0.60,0.60}{##1}}}
\expandafter\def\csname PY@tok@na\endcsname{\def\PY@tc##1{\textcolor[rgb]{0.49,0.56,0.16}{##1}}}
\expandafter\def\csname PY@tok@nt\endcsname{\let\PY@bf=\textbf\def\PY@tc##1{\textcolor[rgb]{0.00,0.50,0.00}{##1}}}
\expandafter\def\csname PY@tok@nd\endcsname{\def\PY@tc##1{\textcolor[rgb]{0.67,0.13,1.00}{##1}}}
\expandafter\def\csname PY@tok@s\endcsname{\def\PY@tc##1{\textcolor[rgb]{0.73,0.13,0.13}{##1}}}
\expandafter\def\csname PY@tok@sd\endcsname{\let\PY@it=\textit\def\PY@tc##1{\textcolor[rgb]{0.73,0.13,0.13}{##1}}}
\expandafter\def\csname PY@tok@si\endcsname{\let\PY@bf=\textbf\def\PY@tc##1{\textcolor[rgb]{0.73,0.40,0.53}{##1}}}
\expandafter\def\csname PY@tok@se\endcsname{\let\PY@bf=\textbf\def\PY@tc##1{\textcolor[rgb]{0.73,0.40,0.13}{##1}}}
\expandafter\def\csname PY@tok@sr\endcsname{\def\PY@tc##1{\textcolor[rgb]{0.73,0.40,0.53}{##1}}}
\expandafter\def\csname PY@tok@ss\endcsname{\def\PY@tc##1{\textcolor[rgb]{0.10,0.09,0.49}{##1}}}
\expandafter\def\csname PY@tok@sx\endcsname{\def\PY@tc##1{\textcolor[rgb]{0.00,0.50,0.00}{##1}}}
\expandafter\def\csname PY@tok@m\endcsname{\def\PY@tc##1{\textcolor[rgb]{0.40,0.40,0.40}{##1}}}
\expandafter\def\csname PY@tok@gh\endcsname{\let\PY@bf=\textbf\def\PY@tc##1{\textcolor[rgb]{0.00,0.00,0.50}{##1}}}
\expandafter\def\csname PY@tok@gu\endcsname{\let\PY@bf=\textbf\def\PY@tc##1{\textcolor[rgb]{0.50,0.00,0.50}{##1}}}
\expandafter\def\csname PY@tok@gd\endcsname{\def\PY@tc##1{\textcolor[rgb]{0.63,0.00,0.00}{##1}}}
\expandafter\def\csname PY@tok@gi\endcsname{\def\PY@tc##1{\textcolor[rgb]{0.00,0.63,0.00}{##1}}}
\expandafter\def\csname PY@tok@gr\endcsname{\def\PY@tc##1{\textcolor[rgb]{1.00,0.00,0.00}{##1}}}
\expandafter\def\csname PY@tok@ge\endcsname{\let\PY@it=\textit}
\expandafter\def\csname PY@tok@gs\endcsname{\let\PY@bf=\textbf}
\expandafter\def\csname PY@tok@gp\endcsname{\let\PY@bf=\textbf\def\PY@tc##1{\textcolor[rgb]{0.00,0.00,0.50}{##1}}}
\expandafter\def\csname PY@tok@go\endcsname{\def\PY@tc##1{\textcolor[rgb]{0.53,0.53,0.53}{##1}}}
\expandafter\def\csname PY@tok@gt\endcsname{\def\PY@tc##1{\textcolor[rgb]{0.00,0.27,0.87}{##1}}}
\expandafter\def\csname PY@tok@err\endcsname{\def\PY@bc##1{\setlength{\fboxsep}{0pt}\fcolorbox[rgb]{1.00,0.00,0.00}{1,1,1}{\strut ##1}}}
\expandafter\def\csname PY@tok@kc\endcsname{\let\PY@bf=\textbf\def\PY@tc##1{\textcolor[rgb]{0.00,0.50,0.00}{##1}}}
\expandafter\def\csname PY@tok@kd\endcsname{\let\PY@bf=\textbf\def\PY@tc##1{\textcolor[rgb]{0.00,0.50,0.00}{##1}}}
\expandafter\def\csname PY@tok@kn\endcsname{\let\PY@bf=\textbf\def\PY@tc##1{\textcolor[rgb]{0.00,0.50,0.00}{##1}}}
\expandafter\def\csname PY@tok@kr\endcsname{\let\PY@bf=\textbf\def\PY@tc##1{\textcolor[rgb]{0.00,0.50,0.00}{##1}}}
\expandafter\def\csname PY@tok@bp\endcsname{\def\PY@tc##1{\textcolor[rgb]{0.00,0.50,0.00}{##1}}}
\expandafter\def\csname PY@tok@fm\endcsname{\def\PY@tc##1{\textcolor[rgb]{0.00,0.00,1.00}{##1}}}
\expandafter\def\csname PY@tok@vc\endcsname{\def\PY@tc##1{\textcolor[rgb]{0.10,0.09,0.49}{##1}}}
\expandafter\def\csname PY@tok@vg\endcsname{\def\PY@tc##1{\textcolor[rgb]{0.10,0.09,0.49}{##1}}}
\expandafter\def\csname PY@tok@vi\endcsname{\def\PY@tc##1{\textcolor[rgb]{0.10,0.09,0.49}{##1}}}
\expandafter\def\csname PY@tok@vm\endcsname{\def\PY@tc##1{\textcolor[rgb]{0.10,0.09,0.49}{##1}}}
\expandafter\def\csname PY@tok@sa\endcsname{\def\PY@tc##1{\textcolor[rgb]{0.73,0.13,0.13}{##1}}}
\expandafter\def\csname PY@tok@sb\endcsname{\def\PY@tc##1{\textcolor[rgb]{0.73,0.13,0.13}{##1}}}
\expandafter\def\csname PY@tok@sc\endcsname{\def\PY@tc##1{\textcolor[rgb]{0.73,0.13,0.13}{##1}}}
\expandafter\def\csname PY@tok@dl\endcsname{\def\PY@tc##1{\textcolor[rgb]{0.73,0.13,0.13}{##1}}}
\expandafter\def\csname PY@tok@s2\endcsname{\def\PY@tc##1{\textcolor[rgb]{0.73,0.13,0.13}{##1}}}
\expandafter\def\csname PY@tok@sh\endcsname{\def\PY@tc##1{\textcolor[rgb]{0.73,0.13,0.13}{##1}}}
\expandafter\def\csname PY@tok@s1\endcsname{\def\PY@tc##1{\textcolor[rgb]{0.73,0.13,0.13}{##1}}}
\expandafter\def\csname PY@tok@mb\endcsname{\def\PY@tc##1{\textcolor[rgb]{0.40,0.40,0.40}{##1}}}
\expandafter\def\csname PY@tok@mf\endcsname{\def\PY@tc##1{\textcolor[rgb]{0.40,0.40,0.40}{##1}}}
\expandafter\def\csname PY@tok@mh\endcsname{\def\PY@tc##1{\textcolor[rgb]{0.40,0.40,0.40}{##1}}}
\expandafter\def\csname PY@tok@mi\endcsname{\def\PY@tc##1{\textcolor[rgb]{0.40,0.40,0.40}{##1}}}
\expandafter\def\csname PY@tok@il\endcsname{\def\PY@tc##1{\textcolor[rgb]{0.40,0.40,0.40}{##1}}}
\expandafter\def\csname PY@tok@mo\endcsname{\def\PY@tc##1{\textcolor[rgb]{0.40,0.40,0.40}{##1}}}
\expandafter\def\csname PY@tok@ch\endcsname{\let\PY@it=\textit\def\PY@tc##1{\textcolor[rgb]{0.25,0.50,0.50}{##1}}}
\expandafter\def\csname PY@tok@cm\endcsname{\let\PY@it=\textit\def\PY@tc##1{\textcolor[rgb]{0.25,0.50,0.50}{##1}}}
\expandafter\def\csname PY@tok@cpf\endcsname{\let\PY@it=\textit\def\PY@tc##1{\textcolor[rgb]{0.25,0.50,0.50}{##1}}}
\expandafter\def\csname PY@tok@c1\endcsname{\let\PY@it=\textit\def\PY@tc##1{\textcolor[rgb]{0.25,0.50,0.50}{##1}}}
\expandafter\def\csname PY@tok@cs\endcsname{\let\PY@it=\textit\def\PY@tc##1{\textcolor[rgb]{0.25,0.50,0.50}{##1}}}

\def\PYZbs{\char`\\}
\def\PYZus{\char`\_}
\def\PYZob{\char`\{}
\def\PYZcb{\char`\}}
\def\PYZca{\char`\^}
\def\PYZam{\char`\&}
\def\PYZlt{\char`\<}
\def\PYZgt{\char`\>}
\def\PYZsh{\char`\#}
\def\PYZpc{\char`\%}
\def\PYZdl{\char`\$}
\def\PYZhy{\char`\-}
\def\PYZsq{\char`\'}
\def\PYZdq{\char`\"}
\def\PYZti{\char`\~}
% for compatibility with earlier versions
\def\PYZat{@}
\def\PYZlb{[}
\def\PYZrb{]}
\makeatother


    % Exact colors from NB
    \definecolor{incolor}{rgb}{0.0, 0.0, 0.5}
    \definecolor{outcolor}{rgb}{0.545, 0.0, 0.0}



    
    % Prevent overflowing lines due to hard-to-break entities
    \sloppy 
    % Setup hyperref package
    \hypersetup{
      breaklinks=true,  % so long urls are correctly broken across lines
      colorlinks=true,
      urlcolor=urlcolor,
      linkcolor=linkcolor,
      citecolor=citecolor,
      }
    % Slightly bigger margins than the latex defaults
    
    \geometry{verbose,tmargin=1in,bmargin=1in,lmargin=1in,rmargin=1in}
    
    

    \begin{document}
    
    
    \maketitle
    
    

    
    \section{Homework 1}\label{homework-1}

\subsubsection{Name: Garret Grant}\label{name-garret-grant}

\subsubsection{Collaborators: N/A}\label{collaborators-na}

    This homework focuses on topics related to basic data types,
collections, and iterations.

I encourage collaborating with your peers, but the final text, code, and
comments in this homework assignment should still be written by you.

Pay special attention to the instructions - should your function
\texttt{print} something or \texttt{return} something?

\section{Submission instructions:}\label{submission-instructions}

\begin{itemize}
\tightlist
\item
  When you're ready to submit, click this "fastforward button":
\end{itemize}

\begin{figure}
\centering
\includegraphics{submission.png}
\caption{image}
\end{figure}

This will reload your functions from \texttt{HW1.py} and rerun all the
cells in your Jupyter notebook. Double check to make sure that your
outputs remain the same as before. - Make a zip file of \texttt{HW1.py}
and \texttt{gimme\_an\_odd\_number.py}. Do \textbf{NOT} change the file
names. - Submit the zip file on Gradescope under "HW1 - Autograder". The
grade you see is the grade you get for the accuracy portion of your code
in problems 2-5 (so no surprises). The style and readability of your
code will be checked by the reader aka human grader. - Convert this
notebook into a pdf file and submit it on GradeScope under "HW1 - PDF".
Make sure the figure in the last part is visible.

    \subsection{Comments and Docstrings}\label{comments-and-docstrings}

You will be graded in part on the quality of your documentation and
explanation of your code. Here's what we expect:

\begin{itemize}
\tightlist
\item
  \textbf{Comments}: Use comments liberally to explain the purpose of
  short snippets of code.
\item
  \textbf{Docstrings}: Functions (and, later, classes) should be
  accompanied by a \emph{docstring}. Briefly, the docstring should
  provide enough information that a user could correctly use your
  function \textbf{\emph{without seeing the code.}} In somewhat more
  detail, the docstring should include the following information:

  \begin{itemize}
  \tightlist
  \item
    One or more sentences describing the overall purpose of the
    function.
  \item
    An explanation of each of the inputs, including what they mean,
    their required data types, and any additional assumptions made about
    them.
  \item
    An explanation of the outputs.
  \end{itemize}
\end{itemize}

We gave you docstrings in this homework, and you should read them all!
In future homeworks, as well as on exams, we will be looking for clear
and informative comments and docstrings.

\subsection{Code Structure}\label{code-structure}

In general, there are many good ways to solve a given problem. However,
just getting the right result isn't enough to guarantee that your code
is of high quality. Check the logic of your solutions to make sure that:

\begin{itemize}
\tightlist
\item
  You aren't making any unnecessary steps, like creating variables you
  don't use.
\item
  You are effectively making use of the tools in the course, especially
  control flow.
\item
  Your code is readable. Each line is short (under 80 characters), and
  doesn't have long tangles of functions or \texttt{()} parentheses.
\end{itemize}

Ok, let's go!

    \begin{Verbatim}[commandchars=\\\{\}]
{\color{incolor}In [{\color{incolor}1}]:} \PY{c+c1}{\PYZsh{} This cell imports your functions defined in HW1.py }
        \PY{k+kn}{from} \PY{n+nn}{HW1} \PY{k}{import} \PY{n}{make\PYZus{}count\PYZus{}dictionary}
        \PY{k+kn}{from} \PY{n+nn}{HW1} \PY{k}{import} \PY{n}{get\PYZus{}triangular\PYZus{}numbers}\PY{p}{,} \PY{n}{get\PYZus{}consonants}\PY{p}{,} \PY{n}{get\PYZus{}list\PYZus{}of\PYZus{}powers}\PY{p}{,} \PY{n}{get\PYZus{}list\PYZus{}of\PYZus{}even\PYZus{}powers}
        \PY{k+kn}{from} \PY{n+nn}{HW1} \PY{k}{import} \PY{n}{random\PYZus{}walk}
\end{Verbatim}


    \section{Problem 1}\label{problem-1}

\subsection{(a)}\label{a}

In the cell below, define a string variable \texttt{s} such that
\texttt{print(s)} prints:

\begin{verbatim}
Tired    : Doing math on your calculator. 
Wired    : Doing math in Python. 
Inspired : Training literal pythons to carry out long division using an abacus.
\end{verbatim}

The potentially tricky part here is dealing with the newlines. You can
choose to use newline characters, or use triple quotes. See:
https://docs.python.org/3/tutorial/introduction.html\#strings.

    \begin{Verbatim}[commandchars=\\\{\}]
{\color{incolor}In [{\color{incolor}2}]:} \PY{c+c1}{\PYZsh{} define s and print it here}
        \PY{n}{s} \PY{o}{=} \PY{l+s+s2}{\PYZdq{}\PYZdq{}\PYZdq{}}\PY{l+s+s2}{Tired    : Doing math on your calculator. }
        \PY{l+s+s2}{Wired    : Doing math in Python. }
        \PY{l+s+s2}{Inspired : Training literal pythons to carry out long division using an abacus.}\PY{l+s+s2}{\PYZdq{}\PYZdq{}\PYZdq{}}
        
        \PY{n+nb}{print}\PY{p}{(}\PY{n}{s}\PY{p}{)}
\end{Verbatim}


    \begin{Verbatim}[commandchars=\\\{\}]
Tired    : Doing math on your calculator. 
Wired    : Doing math in Python. 
Inspired : Training literal pythons to carry out long division using an abacus.

    \end{Verbatim}

    Next, write Python commands which use \texttt{s} to print the specified
outputs. Feel free to use loops and comprehensions; however, keep your
code as concise as possible. Each solution should require at most three
short lines of code.

For full credit, you should minimize the use of positional indexing
(e.g. \texttt{s{[}5:10{]}}) when possible.

\subsection{(b)}\label{b}

Output:

\begin{verbatim}
Tired    
Doing math on your calculator. 
Wired    
Doing math in Python. 
Inspired 
Training literal pythons to carry out long division using an abacus.
\end{verbatim}

    \begin{Verbatim}[commandchars=\\\{\}]
{\color{incolor}In [{\color{incolor}3}]:} \PY{k+kn}{import} \PY{n+nn}{re}
        \PY{k}{for} \PY{n}{line} \PY{o+ow}{in} \PY{n}{re}\PY{o}{.}\PY{n}{split}\PY{p}{(}\PY{l+s+s1}{\PYZsq{}}\PY{l+s+s1}{:|}\PY{l+s+se}{\PYZbs{}n}\PY{l+s+s1}{\PYZsq{}}\PY{p}{,}\PY{n}{s}\PY{p}{)}\PY{p}{:}
            \PY{n+nb}{print}\PY{p}{(}\PY{n}{line}\PY{o}{.}\PY{n}{strip}\PY{p}{(}\PY{p}{)}\PY{p}{)}
\end{Verbatim}


    \begin{Verbatim}[commandchars=\\\{\}]
Tired
Doing math on your calculator.
Wired
Doing math in Python.
Inspired
Training literal pythons to carry out long division using an abacus.

    \end{Verbatim}

    \subsection{(c)}\label{c}

Output:

\begin{verbatim}
Tired
Wired
Inspired
\end{verbatim}

\textbf{Hint}: look at the endings of words. A small amount of
positional indexing might be handy here.

    \begin{Verbatim}[commandchars=\\\{\}]
{\color{incolor}In [{\color{incolor}4}]:} \PY{k}{for} \PY{n}{line} \PY{o+ow}{in} \PY{n}{re}\PY{o}{.}\PY{n}{split}\PY{p}{(}\PY{l+s+s1}{\PYZsq{}}\PY{l+s+s1}{:|}\PY{l+s+se}{\PYZbs{}n}\PY{l+s+s1}{\PYZsq{}}\PY{p}{,}\PY{n}{s}\PY{p}{)}\PY{p}{:}
            \PY{k}{if} \PY{n}{line}\PY{o}{.}\PY{n}{strip}\PY{p}{(}\PY{p}{)}\PY{p}{[}\PY{o}{\PYZhy{}}\PY{l+m+mi}{1}\PY{p}{]} \PY{o}{==} \PY{l+s+s1}{\PYZsq{}}\PY{l+s+s1}{d}\PY{l+s+s1}{\PYZsq{}}\PY{p}{:}
                \PY{n+nb}{print}\PY{p}{(}\PY{n}{line}\PY{o}{.}\PY{n}{strip}\PY{p}{(}\PY{p}{)}\PY{p}{)} 
\end{Verbatim}


    \begin{Verbatim}[commandchars=\\\{\}]
Tired
Wired
Inspired

    \end{Verbatim}

    \subsection{(d)}\label{d}

Output:

\begin{verbatim}
Tired    : Doing math on your calculator. 
Wired    : Doing math in Python. 
\end{verbatim}

\textbf{Hint}: These two lines are shorter than the other one. You are
NOT allowed to use the fact that these are the first two sentences of
the text.

    \begin{Verbatim}[commandchars=\\\{\}]
{\color{incolor}In [{\color{incolor}5}]:} \PY{n}{max\PYZus{}l}\PY{p}{,} \PY{n}{lines} \PY{o}{=} \PY{l+m+mi}{0}\PY{p}{,} \PY{n}{s}\PY{o}{.}\PY{n}{split}\PY{p}{(}\PY{l+s+s1}{\PYZsq{}}\PY{l+s+se}{\PYZbs{}n}\PY{l+s+s1}{\PYZsq{}}\PY{p}{)}
        \PY{k}{for} \PY{n}{line} \PY{o+ow}{in} \PY{n}{lines}\PY{p}{:}
            \PY{n}{max\PYZus{}l} \PY{o}{=} \PY{n+nb}{len}\PY{p}{(}\PY{n}{line}\PY{p}{)} \PY{k}{if} \PY{n+nb}{len}\PY{p}{(}\PY{n}{line}\PY{p}{)} \PY{o}{\PYZgt{}} \PY{n}{max\PYZus{}l} \PY{k}{else} \PY{n}{max\PYZus{}l}
        \PY{k}{for} \PY{n}{line} \PY{o+ow}{in} \PY{n}{lines}\PY{p}{:}
            \PY{k}{if} \PY{n+nb}{len}\PY{p}{(}\PY{n}{line}\PY{p}{)} \PY{o}{\PYZlt{}} \PY{n}{max\PYZus{}l}\PY{p}{:}
                \PY{n+nb}{print}\PY{p}{(}\PY{n}{line}\PY{p}{)} 
\end{Verbatim}


    \begin{Verbatim}[commandchars=\\\{\}]
Tired    : Doing math on your calculator. 
Wired    : Doing math in Python. 

    \end{Verbatim}

    \subsection{(e)}\label{e}

When \texttt{print\_s\_change(s)} is run with the previously defined
\texttt{s}, it should print:

\begin{verbatim}
Tired    : Doing data science on your calculator. 
Wired    : Doing data science in Python. 
Inspired : Training literal pythons to carry out machine learning using an abacus.
\end{verbatim}

\textbf{Hint}: \texttt{str.replace}.

    \begin{Verbatim}[commandchars=\\\{\}]
{\color{incolor}In [{\color{incolor}6}]:} \PY{k}{def} \PY{n+nf}{print\PYZus{}s\PYZus{}change}\PY{p}{(}\PY{n}{string}\PY{p}{)}\PY{p}{:}
            \PY{n+nb}{print}\PY{p}{(}\PY{n}{string}\PY{p}{)}
        
        \PY{n}{print\PYZus{}s\PYZus{}change}\PY{p}{(}\PY{n}{s}\PY{p}{)}
\end{Verbatim}


    \begin{Verbatim}[commandchars=\\\{\}]
Tired    : Doing math on your calculator. 
Wired    : Doing math in Python. 
Inspired : Training literal pythons to carry out long division using an abacus.

    \end{Verbatim}

    \section{\texorpdfstring{Problem 2: Define
\texttt{make\_count\_dictionary} in
HW1.py}{Problem 2: Define make\_count\_dictionary in HW1.py}}\label{problem-2-define-make_count_dictionary-in-hw1.py}

The function \texttt{make\_count\_dictionary} takes a list \texttt{L}
and returns a dictionary \texttt{D} where:

\begin{itemize}
\tightlist
\item
  The \emph{keys} of \texttt{D} are the unique elements of \texttt{L}
  (i.e. each element of \texttt{L} appears only once).
\item
  The value \texttt{D{[}i{]}} is the number of times that \texttt{i}
  appears in list \texttt{L}.
\end{itemize}

Make sure your function has a descriptive docstring and is sufficiently
commented.

Your code should work for lists of strings, lists of integers, and lists
containing both strings and integers.

For example:

\begin{Shaded}
\begin{Highlighting}[]
\CommentTok{# input}
\NormalTok{L }\OperatorTok{=}\NormalTok{ [}\StringTok{"a"}\NormalTok{, }\StringTok{"a"}\NormalTok{, }\StringTok{"b"}\NormalTok{, }\StringTok{"c"}\NormalTok{]}
\CommentTok{# output}
\NormalTok{\{}\StringTok{"a"}\NormalTok{ : }\DecValTok{2}\NormalTok{, }\StringTok{"b"}\NormalTok{ : }\DecValTok{1}\NormalTok{, }\StringTok{"c"}\NormalTok{ : }\DecValTok{1}\NormalTok{\}}
\end{Highlighting}
\end{Shaded}

    \begin{Verbatim}[commandchars=\\\{\}]
{\color{incolor}In [{\color{incolor}7}]:} \PY{c+c1}{\PYZsh{} test your make\PYZus{}count\PYZus{}dictionary here, like this:}
        \PY{c+c1}{\PYZsh{} L = [\PYZdq{}a\PYZdq{}, \PYZdq{}a\PYZdq{}, \PYZdq{}b\PYZdq{}, \PYZdq{}c\PYZdq{}]}
        \PY{c+c1}{\PYZsh{} count\PYZus{}dict = make\PYZus{}count\PYZus{}dictionary(L)}
        \PY{c+c1}{\PYZsh{} print(count\PYZus{}dict)}
        
        \PY{c+c1}{\PYZsh{} come up with other test cases (examples).}
        
        \PY{n}{L} \PY{o}{=} \PY{p}{[}\PY{l+s+s2}{\PYZdq{}}\PY{l+s+s2}{a}\PY{l+s+s2}{\PYZdq{}}\PY{p}{,} \PY{l+s+s2}{\PYZdq{}}\PY{l+s+s2}{a}\PY{l+s+s2}{\PYZdq{}}\PY{p}{,} \PY{l+s+s2}{\PYZdq{}}\PY{l+s+s2}{b}\PY{l+s+s2}{\PYZdq{}}\PY{p}{,} \PY{l+s+s2}{\PYZdq{}}\PY{l+s+s2}{c}\PY{l+s+s2}{\PYZdq{}}\PY{p}{]}
        \PY{n}{count\PYZus{}dict} \PY{o}{=} \PY{n}{make\PYZus{}count\PYZus{}dictionary}\PY{p}{(}\PY{n}{L}\PY{p}{)}
        \PY{n+nb}{print}\PY{p}{(}\PY{n}{count\PYZus{}dict}\PY{p}{)}
        \PY{n}{L} \PY{o}{=} \PY{p}{[}\PY{l+s+s2}{\PYZdq{}}\PY{l+s+s2}{a}\PY{l+s+s2}{\PYZdq{}}\PY{p}{,} \PY{l+s+s2}{\PYZdq{}}\PY{l+s+s2}{a}\PY{l+s+s2}{\PYZdq{}}\PY{p}{,} \PY{l+s+s2}{\PYZdq{}}\PY{l+s+s2}{a}\PY{l+s+s2}{\PYZdq{}}\PY{p}{,} \PY{l+s+s2}{\PYZdq{}}\PY{l+s+s2}{a}\PY{l+s+s2}{\PYZdq{}}\PY{p}{,} \PY{l+m+mi}{1}\PY{p}{,} \PY{l+m+mi}{2} \PY{p}{,}\PY{l+m+mi}{3}\PY{p}{,} \PY{l+m+mi}{4}\PY{p}{,} \PY{l+m+mi}{4}\PY{p}{,} \PY{l+m+mi}{4}\PY{p}{,}\PY{l+s+s2}{\PYZdq{}}\PY{l+s+s2}{b}\PY{l+s+s2}{\PYZdq{}}\PY{p}{,} \PY{l+s+s2}{\PYZdq{}}\PY{l+s+s2}{c}\PY{l+s+s2}{\PYZdq{}}\PY{p}{]}
        \PY{n}{count\PYZus{}dict} \PY{o}{=} \PY{n}{make\PYZus{}count\PYZus{}dictionary}\PY{p}{(}\PY{n}{L}\PY{p}{)}
        \PY{n+nb}{print}\PY{p}{(}\PY{n}{count\PYZus{}dict}\PY{p}{)}
\end{Verbatim}


    \begin{Verbatim}[commandchars=\\\{\}]
\{'a': 2, 'b': 1, 'c': 1\}
\{'a': 4, 1: 1, 2: 1, 3: 1, 4: 3, 'b': 1, 'c': 1\}

    \end{Verbatim}

    \section{\texorpdfstring{Problem 3: Write the script
\texttt{gimme\_an\_odd\_number.py}}{Problem 3: Write the script gimme\_an\_odd\_number.py}}\label{problem-3-write-the-script-gimme_an_odd_number.py}

The \texttt{input()} function allows you to accept typed input from a
user as a string. For example,

\begin{Shaded}
\begin{Highlighting}[]
\NormalTok{x }\OperatorTok{=} \BuiltInTok{input}\NormalTok{(}\StringTok{"Please enter an integer."}\NormalTok{)}
\CommentTok{# user types 7}
\NormalTok{x}
\CommentTok{# output }
\CommentTok{'7'}
\end{Highlighting}
\end{Shaded}

When \texttt{gimme\_an\_odd\_number.py} is run, it prompts to
\texttt{"Please\ enter\ an\ integer."}. If the user inputs an even
integer, the code should re-prompt them with the same message. If the
user has entered an odd integer, the function should print a list of all
numbers that the user has given so far.

You may assume that the user will only input strings of integers such as
\texttt{"3"} or \texttt{"42"}.

\emph{Hint}: Try \texttt{while} and associated tools.

\emph{Hint}: Which built-in Python function
(\url{https://docs.python.org/3/library/functions.html}) can turn string
\texttt{"3"} to integer \texttt{3}?

\subsubsection{Example}\label{example}

\begin{Shaded}
\begin{Highlighting}[]

\CommentTok{# run gimme_an_odd_number.py}

\OperatorTok{>}\NormalTok{ Please enter an integer.}\DecValTok{6}
\OperatorTok{>}\NormalTok{ Please enter an integer.}\DecValTok{8}
\OperatorTok{>}\NormalTok{ Please enter an integer.}\DecValTok{4}
\OperatorTok{>}\NormalTok{ Please enter an integer.}\DecValTok{9}
\OperatorTok{>}\NormalTok{ [}\DecValTok{6}\NormalTok{, }\DecValTok{8}\NormalTok{, }\DecValTok{4}\NormalTok{, }\DecValTok{9}\NormalTok{]}
\end{Highlighting}
\end{Shaded}

    \begin{Verbatim}[commandchars=\\\{\}]
{\color{incolor}In [{\color{incolor}8}]:} \PY{k}{def} \PY{n+nf}{gimme\PYZus{}an\PYZus{}odd\PYZus{}number}\PY{p}{(}\PY{p}{)}\PY{p}{:}
            \PY{n}{numbers} \PY{o}{=} \PY{p}{[}\PY{p}{]}
            \PY{k}{while} \PY{k+kc}{True}\PY{p}{:}
                \PY{n}{number} \PY{o}{=} \PY{n+nb}{int}\PY{p}{(}\PY{n+nb}{input}\PY{p}{(}\PY{l+s+s2}{\PYZdq{}}\PY{l+s+s2}{Please enter an integer: }\PY{l+s+s2}{\PYZdq{}}\PY{p}{)}\PY{p}{)}
                \PY{n}{numbers}\PY{o}{.}\PY{n}{append}\PY{p}{(}\PY{n}{number}\PY{p}{)}
                \PY{k}{if} \PY{n}{number} \PY{o}{\PYZpc{}} \PY{l+m+mi}{2} \PY{o}{==} \PY{l+m+mi}{0}\PY{p}{:}
                    \PY{k}{continue}
                \PY{k}{else}\PY{p}{:}
                    \PY{n+nb}{print}\PY{p}{(}\PY{n}{numbers}\PY{p}{)}
                \PY{k}{break}
                
        \PY{n}{gimme\PYZus{}an\PYZus{}odd\PYZus{}number}\PY{p}{(}\PY{p}{)}
\end{Verbatim}


    \begin{Verbatim}[commandchars=\\\{\}]
Please enter an integer: 3
[3]

    \end{Verbatim}

    \section{Problem 4}\label{problem-4}

Write list comprehensions which produce the specified list.

For problems c and d, you can either write a nested list comprehension
(https://docs.python.org/3/tutorial/datastructures.html\#nested-list-comprehensions),
or use one loop + list comprehension inside. I would recommend the
following steps: - write something that works with two loops (nested
loops) - condense the inner loop into a list comprehension - try turning
the whole thing into a nested list comprehension

    \subsection{\texorpdfstring{(a) Define \texttt{get\_triangular\_numbers}
in
HW1.py}{(a) Define get\_triangular\_numbers in HW1.py}}\label{a-define-get_triangular_numbers-in-hw1.py}

The \texttt{k}th triangular number
(\url{https://en.wikipedia.org/wiki/Triangular_number}) is the sum of
natural numbers up to and including \texttt{k}. Write
\texttt{get\_triangular\_numbers} such that for a given \texttt{k}, it
returns a list of the first \texttt{k} triangular numbers.

For example, the sixth triangular number is

\[1+2+3+4+5+6 = 21,\]

and running \texttt{get\_triangular\_numbers} with an argument of
\texttt{k=6} should output \texttt{{[}1,\ 3,\ 6,\ 10,\ 15,\ 21{]}}. You
function should have a docstring.

    \begin{Verbatim}[commandchars=\\\{\}]
{\color{incolor}In [{\color{incolor}9}]:} \PY{c+c1}{\PYZsh{} test your get\PYZus{}triangular\PYZus{}numbers here}
        \PY{n}{k} \PY{o}{=} \PY{l+m+mi}{3}
        \PY{n+nb}{print}\PY{p}{(}\PY{n}{get\PYZus{}triangular\PYZus{}numbers}\PY{p}{(}\PY{n}{k}\PY{p}{)}\PY{p}{)}
        
        \PY{n}{k} \PY{o}{=} \PY{l+m+mi}{10}
        \PY{n+nb}{print}\PY{p}{(}\PY{n}{get\PYZus{}triangular\PYZus{}numbers}\PY{p}{(}\PY{n}{k}\PY{p}{)}\PY{p}{)}
\end{Verbatim}


    \begin{Verbatim}[commandchars=\\\{\}]
[1, 3, 6]
[1, 3, 6, 10, 15, 21, 28, 36, 45, 55]

    \end{Verbatim}

    \subsection{\texorpdfstring{(b) Define \texttt{get\_consonants} in
HW1.py}{(b) Define get\_consonants in HW1.py}}\label{b-define-get_consonants-in-hw1.py}

The function \texttt{get\_consonants} taks a string \texttt{s} as an
input, and returns a list of the letters in \texttt{s} \textbf{except
for vowels, spaces, commas, and periods.} For the purposes of this
example, an English vowel is any of the letters
\texttt{{[}"a",\ "e",\ "i",\ "o",\ "u"{]}}. For example:

\begin{Shaded}
\begin{Highlighting}[]
\NormalTok{s }\OperatorTok{=} \StringTok{"make it so, number one"}
\BuiltInTok{print}\NormalTok{(get_consonants(s))}
\NormalTok{[}\StringTok{"m"}\NormalTok{, }\StringTok{"k"}\NormalTok{, }\StringTok{"t"}\NormalTok{, }\StringTok{"s"}\NormalTok{, }\StringTok{"n"}\NormalTok{, }\StringTok{"m"}\NormalTok{, }\StringTok{"b"}\NormalTok{, }\StringTok{"r"}\NormalTok{, }\StringTok{"n"}\NormalTok{]}
\end{Highlighting}
\end{Shaded}

\emph{Hint:} Consider the following code:

\begin{Shaded}
\begin{Highlighting}[]
\NormalTok{l }\OperatorTok{=} \StringTok{"a"}
\NormalTok{l }\KeywordTok{not} \KeywordTok{in}\NormalTok{ [}\StringTok{"e"}\NormalTok{, }\StringTok{"w"}\NormalTok{]}
\end{Highlighting}
\end{Shaded}

    \begin{Verbatim}[commandchars=\\\{\}]
{\color{incolor}In [{\color{incolor}10}]:} \PY{c+c1}{\PYZsh{} test your get\PYZus{}consonants here}
         \PY{n}{s} \PY{o}{=} \PY{l+s+s2}{\PYZdq{}}\PY{l+s+s2}{make it so, number one}\PY{l+s+s2}{\PYZdq{}}
         \PY{n+nb}{print}\PY{p}{(}\PY{n}{get\PYZus{}consonants}\PY{p}{(}\PY{n}{s}\PY{p}{)}\PY{p}{)}
\end{Verbatim}


    \begin{Verbatim}[commandchars=\\\{\}]
['m', 'k', 't', 's', 'n', 'm', 'b', 'r', 'n']

    \end{Verbatim}

    \subsection{\texorpdfstring{(c) Define \texttt{get\_list\_of\_powers} in
HW1.py}{(c) Define get\_list\_of\_powers in HW1.py}}\label{c-define-get_list_of_powers-in-hw1.py}

The function \texttt{get\_list\_of\_powers} takes in a list \texttt{X}
and integer \texttt{k} and returns a list \texttt{L} whose elements are
themselves lists. The \texttt{i}th element of \texttt{L} contains the
powers of \texttt{X{[}i{]}} from \texttt{0} to \texttt{k}.

For example, running \texttt{get\_list\_of\_powers} with inputs
\texttt{X\ =\ {[}5,\ 6,\ 7{]}} and \texttt{k\ =\ 2} will return
\texttt{{[}{[}1,\ 5,\ 25{]},\ {[}1,\ 6,\ 36{]},\ {[}1,\ 7,\ 49{]}{]}}.

    \begin{Verbatim}[commandchars=\\\{\}]
{\color{incolor}In [{\color{incolor}11}]:} \PY{c+c1}{\PYZsh{} test your get\PYZus{}list\PYZus{}of\PYZus{}powers here}
         \PY{n+nb}{print}\PY{p}{(}\PY{n}{get\PYZus{}list\PYZus{}of\PYZus{}powers}\PY{p}{(}\PY{p}{[}\PY{l+m+mi}{5}\PY{p}{,}\PY{l+m+mi}{6}\PY{p}{,}\PY{l+m+mi}{7}\PY{p}{]}\PY{p}{,} \PY{l+m+mi}{2}\PY{p}{)}\PY{p}{)}
\end{Verbatim}


    \begin{Verbatim}[commandchars=\\\{\}]
[[1, 5, 25], [1, 6, 36], [1, 7, 49]]

    \end{Verbatim}

    \subsection{\texorpdfstring{(d) Define
\texttt{get\_list\_of\_even\_powers} in
HW1.py}{(d) Define get\_list\_of\_even\_powers in HW1.py}}\label{d-define-get_list_of_even_powers-in-hw1.py}

As in \textbf{(c)}, the function \texttt{get\_list\_of\_even\_powers}
takes in a list \texttt{X} and inter \texttt{k}, and returns a list
\texttt{L} whose elements are themselves lists. But now \texttt{L}
includes only even powers of elements of \texttt{X}. For example,
running \texttt{get\_list\_of\_even\_powers} with inputs
\texttt{X\ =\ {[}5,\ 6,\ 7{]}} and \texttt{k\ =\ 8} should return
\texttt{{[}{[}1,\ 25,\ 625,\ 15625,\ 390625{]},\ \ {[}1,\ 36,\ 1296,\ 46656,\ 1679616{]},\ \ {[}1,\ 49,\ 2401,\ 117649,\ 5764801{]}{]}}.

    \begin{Verbatim}[commandchars=\\\{\}]
{\color{incolor}In [{\color{incolor}12}]:} \PY{c+c1}{\PYZsh{} test your get\PYZus{}list\PYZus{}of\PYZus{}even\PYZus{}powers here}
         \PY{n+nb}{print}\PY{p}{(}\PY{n}{get\PYZus{}list\PYZus{}of\PYZus{}even\PYZus{}powers}\PY{p}{(}\PY{p}{[}\PY{l+m+mi}{5}\PY{p}{,}\PY{l+m+mi}{6}\PY{p}{,}\PY{l+m+mi}{7}\PY{p}{]}\PY{p}{,} \PY{l+m+mi}{8}\PY{p}{)}\PY{p}{)}
\end{Verbatim}


    \begin{Verbatim}[commandchars=\\\{\}]
[[1, 25, 625, 15625, 390625], [1, 36, 1296, 46656, 1679616], [1, 49, 2401, 117649, 5764801]]

    \end{Verbatim}

    \section{\texorpdfstring{Problem 5: Define \texttt{random\_walk} in
HW1.py}{Problem 5: Define random\_walk in HW1.py}}\label{problem-5-define-random_walk-in-hw1.py}

In this problem, we'll simulate the \emph{simple random walk}, perhaps
the most important discrete-time stochastic process. Random walks are
commonly used to model phenomena in physics, chemistry, biology, and
finance. In the simple random walk, at each timestep we flip a fair
coin. If heads, we move foward one step; if tails, we move backwards.
Let "forwards" be represented by positive integers, and "backwards" be
represented by negative integers. For example, if we are currently three
steps backwards from the starting point, our position is \texttt{-3}.

Write \texttt{random\_walk} to simulate a random walk. Your function
should:

\begin{itemize}
\tightlist
\item
  Take an upper and lower bound as inputs.
\item
  Return three variables \texttt{pos}, \texttt{positions},
  \texttt{steps}, in that order.
\item
  \texttt{pos} is an integer, and indicates the walk's final position at
  termination.
\item
  \texttt{positions} is a list of integers, and it is a log of the
  position of the walk at each time step.
\item
  \texttt{steps} is a list of integers, and it is a log of the results
  of the coin flips.
\item
  Please read the provided docstrings for more details on what the
  autograder expects.
\end{itemize}

When the walk reaches the upper or lower bound, print a message such as
\texttt{Upper\ bound\ at\ 3\ reached} and terminate the walk.

Your code should include at least one instance of an \texttt{elif}
statement and at least one instance of a \texttt{break} statement.

    \textbf{Hint} To simulate a fair coin toss, try running the following
cell multiple times. You don't have to use \texttt{"heads"} and
\texttt{"tails"} when using this function -\/- can you think of a more
useful choice set?

    \begin{Verbatim}[commandchars=\\\{\}]
{\color{incolor}In [{\color{incolor}13}]:} \PY{k+kn}{import} \PY{n+nn}{random}
         \PY{k+kn}{from} \PY{n+nn}{matplotlib} \PY{k}{import} \PY{n}{pyplot} \PY{k}{as} \PY{n}{plt}
\end{Verbatim}


    \begin{Verbatim}[commandchars=\\\{\}]
{\color{incolor}In [{\color{incolor}14}]:} \PY{k}{for} \PY{n}{\PYZus{}} \PY{o+ow}{in} \PY{n+nb}{range}\PY{p}{(}\PY{l+m+mi}{10}\PY{p}{)}\PY{p}{:}
             \PY{n}{x} \PY{o}{=} \PY{n}{random}\PY{o}{.}\PY{n}{choice}\PY{p}{(}\PY{p}{[}\PY{l+s+s2}{\PYZdq{}}\PY{l+s+s2}{heads}\PY{l+s+s2}{\PYZdq{}}\PY{p}{,}\PY{l+s+s2}{\PYZdq{}}\PY{l+s+s2}{tails}\PY{l+s+s2}{\PYZdq{}}\PY{p}{]}\PY{p}{)}
             \PY{n+nb}{print}\PY{p}{(}\PY{n}{x}\PY{p}{)}
\end{Verbatim}


    \begin{Verbatim}[commandchars=\\\{\}]
tails
heads
heads
heads
heads
heads
heads
heads
heads
tails

    \end{Verbatim}

    \begin{Verbatim}[commandchars=\\\{\}]
{\color{incolor}In [{\color{incolor}15}]:} \PY{c+c1}{\PYZsh{} test your random\PYZus{}walk here}
         \PY{n}{pos}\PY{p}{,} \PY{n}{positions}\PY{p}{,} \PY{n}{steps} \PY{o}{=} \PY{n}{random\PYZus{}walk}\PY{p}{(}\PY{l+m+mi}{3}\PY{p}{,} \PY{o}{\PYZhy{}}\PY{l+m+mi}{3}\PY{p}{)}
         \PY{n+nb}{print}\PY{p}{(}\PY{n}{pos}\PY{p}{,} \PY{n}{positions}\PY{p}{,} \PY{n}{steps}\PY{p}{)}
\end{Verbatim}


    \begin{Verbatim}[commandchars=\\\{\}]
-3 [-1, -2, -3] [-1, -1, -1]

    \end{Verbatim}

    Finally, you might be interested in visualizing the walk. Run the
following cell to produce a plot. When the bounds are set very large,
the resulting visualization can be quite intriguing and attractive. It
is not necessary for you to understand the syntax of these commands at
this stage.

    \begin{Verbatim}[commandchars=\\\{\}]
{\color{incolor}In [{\color{incolor}20}]:} \PY{n}{pos}\PY{p}{,} \PY{n}{positions}\PY{p}{,} \PY{n}{steps} \PY{o}{=} \PY{n}{random\PYZus{}walk}\PY{p}{(}\PY{l+m+mi}{5000}\PY{p}{,} \PY{o}{\PYZhy{}}\PY{l+m+mi}{5000}\PY{p}{)}
         
         \PY{n}{plt}\PY{o}{.}\PY{n}{figure}\PY{p}{(}\PY{n}{figsize}\PY{o}{=}\PY{p}{(}\PY{l+m+mi}{12}\PY{p}{,} \PY{l+m+mi}{8}\PY{p}{)}\PY{p}{)}
         \PY{n}{plt}\PY{o}{.}\PY{n}{plot}\PY{p}{(}\PY{n}{positions}\PY{p}{)}
         \PY{n}{plt}\PY{o}{.}\PY{n}{xlabel}\PY{p}{(}\PY{l+s+s1}{\PYZsq{}}\PY{l+s+s1}{Timestep}\PY{l+s+s1}{\PYZsq{}}\PY{p}{)}
         \PY{n}{plt}\PY{o}{.}\PY{n}{ylabel}\PY{p}{(}\PY{l+s+s1}{\PYZsq{}}\PY{l+s+s1}{Position}\PY{l+s+s1}{\PYZsq{}}\PY{p}{)}
         \PY{n}{plt}\PY{o}{.}\PY{n}{title}\PY{p}{(}\PY{l+s+s1}{\PYZsq{}}\PY{l+s+s1}{Random Walk}\PY{l+s+s1}{\PYZsq{}}\PY{p}{)}
         \PY{n}{plt}\PY{o}{.}\PY{n}{show}\PY{p}{(}\PY{p}{)}
         \PY{c+c1}{\PYZsh{} yall 5000, trying to break my computer, chillll}
\end{Verbatim}


    \begin{center}
    \adjustimage{max size={0.9\linewidth}{0.9\paperheight}}{output_33_0.png}
    \end{center}
    { \hspace*{\fill} \\}
    

    % Add a bibliography block to the postdoc
    
    
    
    \end{document}
