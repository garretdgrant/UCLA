
% Default to the notebook output style

    


% Inherit from the specified cell style.




    
\documentclass[11pt]{article}

    
    
    \usepackage[T1]{fontenc}
    % Nicer default font (+ math font) than Computer Modern for most use cases
    \usepackage{mathpazo}

    % Basic figure setup, for now with no caption control since it's done
    % automatically by Pandoc (which extracts ![](path) syntax from Markdown).
    \usepackage{graphicx}
    % We will generate all images so they have a width \maxwidth. This means
    % that they will get their normal width if they fit onto the page, but
    % are scaled down if they would overflow the margins.
    \makeatletter
    \def\maxwidth{\ifdim\Gin@nat@width>\linewidth\linewidth
    \else\Gin@nat@width\fi}
    \makeatother
    \let\Oldincludegraphics\includegraphics
    % Set max figure width to be 80% of text width, for now hardcoded.
    \renewcommand{\includegraphics}[1]{\Oldincludegraphics[width=.8\maxwidth]{#1}}
    % Ensure that by default, figures have no caption (until we provide a
    % proper Figure object with a Caption API and a way to capture that
    % in the conversion process - todo).
    \usepackage{caption}
    \DeclareCaptionLabelFormat{nolabel}{}
    \captionsetup{labelformat=nolabel}

    \usepackage{adjustbox} % Used to constrain images to a maximum size 
    \usepackage{xcolor} % Allow colors to be defined
    \usepackage{enumerate} % Needed for markdown enumerations to work
    \usepackage{geometry} % Used to adjust the document margins
    \usepackage{amsmath} % Equations
    \usepackage{amssymb} % Equations
    \usepackage{textcomp} % defines textquotesingle
    % Hack from http://tex.stackexchange.com/a/47451/13684:
    \AtBeginDocument{%
        \def\PYZsq{\textquotesingle}% Upright quotes in Pygmentized code
    }
    \usepackage{upquote} % Upright quotes for verbatim code
    \usepackage{eurosym} % defines \euro
    \usepackage[mathletters]{ucs} % Extended unicode (utf-8) support
    \usepackage[utf8x]{inputenc} % Allow utf-8 characters in the tex document
    \usepackage{fancyvrb} % verbatim replacement that allows latex
    \usepackage{grffile} % extends the file name processing of package graphics 
                         % to support a larger range 
    % The hyperref package gives us a pdf with properly built
    % internal navigation ('pdf bookmarks' for the table of contents,
    % internal cross-reference links, web links for URLs, etc.)
    \usepackage{hyperref}
    \usepackage{longtable} % longtable support required by pandoc >1.10
    \usepackage{booktabs}  % table support for pandoc > 1.12.2
    \usepackage[inline]{enumitem} % IRkernel/repr support (it uses the enumerate* environment)
    \usepackage[normalem]{ulem} % ulem is needed to support strikethroughs (\sout)
                                % normalem makes italics be italics, not underlines
    

    
    
    % Colors for the hyperref package
    \definecolor{urlcolor}{rgb}{0,.145,.698}
    \definecolor{linkcolor}{rgb}{.71,0.21,0.01}
    \definecolor{citecolor}{rgb}{.12,.54,.11}

    % ANSI colors
    \definecolor{ansi-black}{HTML}{3E424D}
    \definecolor{ansi-black-intense}{HTML}{282C36}
    \definecolor{ansi-red}{HTML}{E75C58}
    \definecolor{ansi-red-intense}{HTML}{B22B31}
    \definecolor{ansi-green}{HTML}{00A250}
    \definecolor{ansi-green-intense}{HTML}{007427}
    \definecolor{ansi-yellow}{HTML}{DDB62B}
    \definecolor{ansi-yellow-intense}{HTML}{B27D12}
    \definecolor{ansi-blue}{HTML}{208FFB}
    \definecolor{ansi-blue-intense}{HTML}{0065CA}
    \definecolor{ansi-magenta}{HTML}{D160C4}
    \definecolor{ansi-magenta-intense}{HTML}{A03196}
    \definecolor{ansi-cyan}{HTML}{60C6C8}
    \definecolor{ansi-cyan-intense}{HTML}{258F8F}
    \definecolor{ansi-white}{HTML}{C5C1B4}
    \definecolor{ansi-white-intense}{HTML}{A1A6B2}

    % commands and environments needed by pandoc snippets
    % extracted from the output of `pandoc -s`
    \providecommand{\tightlist}{%
      \setlength{\itemsep}{0pt}\setlength{\parskip}{0pt}}
    \DefineVerbatimEnvironment{Highlighting}{Verbatim}{commandchars=\\\{\}}
    % Add ',fontsize=\small' for more characters per line
    \newenvironment{Shaded}{}{}
    \newcommand{\KeywordTok}[1]{\textcolor[rgb]{0.00,0.44,0.13}{\textbf{{#1}}}}
    \newcommand{\DataTypeTok}[1]{\textcolor[rgb]{0.56,0.13,0.00}{{#1}}}
    \newcommand{\DecValTok}[1]{\textcolor[rgb]{0.25,0.63,0.44}{{#1}}}
    \newcommand{\BaseNTok}[1]{\textcolor[rgb]{0.25,0.63,0.44}{{#1}}}
    \newcommand{\FloatTok}[1]{\textcolor[rgb]{0.25,0.63,0.44}{{#1}}}
    \newcommand{\CharTok}[1]{\textcolor[rgb]{0.25,0.44,0.63}{{#1}}}
    \newcommand{\StringTok}[1]{\textcolor[rgb]{0.25,0.44,0.63}{{#1}}}
    \newcommand{\CommentTok}[1]{\textcolor[rgb]{0.38,0.63,0.69}{\textit{{#1}}}}
    \newcommand{\OtherTok}[1]{\textcolor[rgb]{0.00,0.44,0.13}{{#1}}}
    \newcommand{\AlertTok}[1]{\textcolor[rgb]{1.00,0.00,0.00}{\textbf{{#1}}}}
    \newcommand{\FunctionTok}[1]{\textcolor[rgb]{0.02,0.16,0.49}{{#1}}}
    \newcommand{\RegionMarkerTok}[1]{{#1}}
    \newcommand{\ErrorTok}[1]{\textcolor[rgb]{1.00,0.00,0.00}{\textbf{{#1}}}}
    \newcommand{\NormalTok}[1]{{#1}}
    
    % Additional commands for more recent versions of Pandoc
    \newcommand{\ConstantTok}[1]{\textcolor[rgb]{0.53,0.00,0.00}{{#1}}}
    \newcommand{\SpecialCharTok}[1]{\textcolor[rgb]{0.25,0.44,0.63}{{#1}}}
    \newcommand{\VerbatimStringTok}[1]{\textcolor[rgb]{0.25,0.44,0.63}{{#1}}}
    \newcommand{\SpecialStringTok}[1]{\textcolor[rgb]{0.73,0.40,0.53}{{#1}}}
    \newcommand{\ImportTok}[1]{{#1}}
    \newcommand{\DocumentationTok}[1]{\textcolor[rgb]{0.73,0.13,0.13}{\textit{{#1}}}}
    \newcommand{\AnnotationTok}[1]{\textcolor[rgb]{0.38,0.63,0.69}{\textbf{\textit{{#1}}}}}
    \newcommand{\CommentVarTok}[1]{\textcolor[rgb]{0.38,0.63,0.69}{\textbf{\textit{{#1}}}}}
    \newcommand{\VariableTok}[1]{\textcolor[rgb]{0.10,0.09,0.49}{{#1}}}
    \newcommand{\ControlFlowTok}[1]{\textcolor[rgb]{0.00,0.44,0.13}{\textbf{{#1}}}}
    \newcommand{\OperatorTok}[1]{\textcolor[rgb]{0.40,0.40,0.40}{{#1}}}
    \newcommand{\BuiltInTok}[1]{{#1}}
    \newcommand{\ExtensionTok}[1]{{#1}}
    \newcommand{\PreprocessorTok}[1]{\textcolor[rgb]{0.74,0.48,0.00}{{#1}}}
    \newcommand{\AttributeTok}[1]{\textcolor[rgb]{0.49,0.56,0.16}{{#1}}}
    \newcommand{\InformationTok}[1]{\textcolor[rgb]{0.38,0.63,0.69}{\textbf{\textit{{#1}}}}}
    \newcommand{\WarningTok}[1]{\textcolor[rgb]{0.38,0.63,0.69}{\textbf{\textit{{#1}}}}}
    
    
    % Define a nice break command that doesn't care if a line doesn't already
    % exist.
    \def\br{\hspace*{\fill} \\* }
    % Math Jax compatability definitions
    \def\gt{>}
    \def\lt{<}
    % Document parameters
    \title{HW4}
    
    
    

    % Pygments definitions
    
\makeatletter
\def\PY@reset{\let\PY@it=\relax \let\PY@bf=\relax%
    \let\PY@ul=\relax \let\PY@tc=\relax%
    \let\PY@bc=\relax \let\PY@ff=\relax}
\def\PY@tok#1{\csname PY@tok@#1\endcsname}
\def\PY@toks#1+{\ifx\relax#1\empty\else%
    \PY@tok{#1}\expandafter\PY@toks\fi}
\def\PY@do#1{\PY@bc{\PY@tc{\PY@ul{%
    \PY@it{\PY@bf{\PY@ff{#1}}}}}}}
\def\PY#1#2{\PY@reset\PY@toks#1+\relax+\PY@do{#2}}

\expandafter\def\csname PY@tok@w\endcsname{\def\PY@tc##1{\textcolor[rgb]{0.73,0.73,0.73}{##1}}}
\expandafter\def\csname PY@tok@c\endcsname{\let\PY@it=\textit\def\PY@tc##1{\textcolor[rgb]{0.25,0.50,0.50}{##1}}}
\expandafter\def\csname PY@tok@cp\endcsname{\def\PY@tc##1{\textcolor[rgb]{0.74,0.48,0.00}{##1}}}
\expandafter\def\csname PY@tok@k\endcsname{\let\PY@bf=\textbf\def\PY@tc##1{\textcolor[rgb]{0.00,0.50,0.00}{##1}}}
\expandafter\def\csname PY@tok@kp\endcsname{\def\PY@tc##1{\textcolor[rgb]{0.00,0.50,0.00}{##1}}}
\expandafter\def\csname PY@tok@kt\endcsname{\def\PY@tc##1{\textcolor[rgb]{0.69,0.00,0.25}{##1}}}
\expandafter\def\csname PY@tok@o\endcsname{\def\PY@tc##1{\textcolor[rgb]{0.40,0.40,0.40}{##1}}}
\expandafter\def\csname PY@tok@ow\endcsname{\let\PY@bf=\textbf\def\PY@tc##1{\textcolor[rgb]{0.67,0.13,1.00}{##1}}}
\expandafter\def\csname PY@tok@nb\endcsname{\def\PY@tc##1{\textcolor[rgb]{0.00,0.50,0.00}{##1}}}
\expandafter\def\csname PY@tok@nf\endcsname{\def\PY@tc##1{\textcolor[rgb]{0.00,0.00,1.00}{##1}}}
\expandafter\def\csname PY@tok@nc\endcsname{\let\PY@bf=\textbf\def\PY@tc##1{\textcolor[rgb]{0.00,0.00,1.00}{##1}}}
\expandafter\def\csname PY@tok@nn\endcsname{\let\PY@bf=\textbf\def\PY@tc##1{\textcolor[rgb]{0.00,0.00,1.00}{##1}}}
\expandafter\def\csname PY@tok@ne\endcsname{\let\PY@bf=\textbf\def\PY@tc##1{\textcolor[rgb]{0.82,0.25,0.23}{##1}}}
\expandafter\def\csname PY@tok@nv\endcsname{\def\PY@tc##1{\textcolor[rgb]{0.10,0.09,0.49}{##1}}}
\expandafter\def\csname PY@tok@no\endcsname{\def\PY@tc##1{\textcolor[rgb]{0.53,0.00,0.00}{##1}}}
\expandafter\def\csname PY@tok@nl\endcsname{\def\PY@tc##1{\textcolor[rgb]{0.63,0.63,0.00}{##1}}}
\expandafter\def\csname PY@tok@ni\endcsname{\let\PY@bf=\textbf\def\PY@tc##1{\textcolor[rgb]{0.60,0.60,0.60}{##1}}}
\expandafter\def\csname PY@tok@na\endcsname{\def\PY@tc##1{\textcolor[rgb]{0.49,0.56,0.16}{##1}}}
\expandafter\def\csname PY@tok@nt\endcsname{\let\PY@bf=\textbf\def\PY@tc##1{\textcolor[rgb]{0.00,0.50,0.00}{##1}}}
\expandafter\def\csname PY@tok@nd\endcsname{\def\PY@tc##1{\textcolor[rgb]{0.67,0.13,1.00}{##1}}}
\expandafter\def\csname PY@tok@s\endcsname{\def\PY@tc##1{\textcolor[rgb]{0.73,0.13,0.13}{##1}}}
\expandafter\def\csname PY@tok@sd\endcsname{\let\PY@it=\textit\def\PY@tc##1{\textcolor[rgb]{0.73,0.13,0.13}{##1}}}
\expandafter\def\csname PY@tok@si\endcsname{\let\PY@bf=\textbf\def\PY@tc##1{\textcolor[rgb]{0.73,0.40,0.53}{##1}}}
\expandafter\def\csname PY@tok@se\endcsname{\let\PY@bf=\textbf\def\PY@tc##1{\textcolor[rgb]{0.73,0.40,0.13}{##1}}}
\expandafter\def\csname PY@tok@sr\endcsname{\def\PY@tc##1{\textcolor[rgb]{0.73,0.40,0.53}{##1}}}
\expandafter\def\csname PY@tok@ss\endcsname{\def\PY@tc##1{\textcolor[rgb]{0.10,0.09,0.49}{##1}}}
\expandafter\def\csname PY@tok@sx\endcsname{\def\PY@tc##1{\textcolor[rgb]{0.00,0.50,0.00}{##1}}}
\expandafter\def\csname PY@tok@m\endcsname{\def\PY@tc##1{\textcolor[rgb]{0.40,0.40,0.40}{##1}}}
\expandafter\def\csname PY@tok@gh\endcsname{\let\PY@bf=\textbf\def\PY@tc##1{\textcolor[rgb]{0.00,0.00,0.50}{##1}}}
\expandafter\def\csname PY@tok@gu\endcsname{\let\PY@bf=\textbf\def\PY@tc##1{\textcolor[rgb]{0.50,0.00,0.50}{##1}}}
\expandafter\def\csname PY@tok@gd\endcsname{\def\PY@tc##1{\textcolor[rgb]{0.63,0.00,0.00}{##1}}}
\expandafter\def\csname PY@tok@gi\endcsname{\def\PY@tc##1{\textcolor[rgb]{0.00,0.63,0.00}{##1}}}
\expandafter\def\csname PY@tok@gr\endcsname{\def\PY@tc##1{\textcolor[rgb]{1.00,0.00,0.00}{##1}}}
\expandafter\def\csname PY@tok@ge\endcsname{\let\PY@it=\textit}
\expandafter\def\csname PY@tok@gs\endcsname{\let\PY@bf=\textbf}
\expandafter\def\csname PY@tok@gp\endcsname{\let\PY@bf=\textbf\def\PY@tc##1{\textcolor[rgb]{0.00,0.00,0.50}{##1}}}
\expandafter\def\csname PY@tok@go\endcsname{\def\PY@tc##1{\textcolor[rgb]{0.53,0.53,0.53}{##1}}}
\expandafter\def\csname PY@tok@gt\endcsname{\def\PY@tc##1{\textcolor[rgb]{0.00,0.27,0.87}{##1}}}
\expandafter\def\csname PY@tok@err\endcsname{\def\PY@bc##1{\setlength{\fboxsep}{0pt}\fcolorbox[rgb]{1.00,0.00,0.00}{1,1,1}{\strut ##1}}}
\expandafter\def\csname PY@tok@kc\endcsname{\let\PY@bf=\textbf\def\PY@tc##1{\textcolor[rgb]{0.00,0.50,0.00}{##1}}}
\expandafter\def\csname PY@tok@kd\endcsname{\let\PY@bf=\textbf\def\PY@tc##1{\textcolor[rgb]{0.00,0.50,0.00}{##1}}}
\expandafter\def\csname PY@tok@kn\endcsname{\let\PY@bf=\textbf\def\PY@tc##1{\textcolor[rgb]{0.00,0.50,0.00}{##1}}}
\expandafter\def\csname PY@tok@kr\endcsname{\let\PY@bf=\textbf\def\PY@tc##1{\textcolor[rgb]{0.00,0.50,0.00}{##1}}}
\expandafter\def\csname PY@tok@bp\endcsname{\def\PY@tc##1{\textcolor[rgb]{0.00,0.50,0.00}{##1}}}
\expandafter\def\csname PY@tok@fm\endcsname{\def\PY@tc##1{\textcolor[rgb]{0.00,0.00,1.00}{##1}}}
\expandafter\def\csname PY@tok@vc\endcsname{\def\PY@tc##1{\textcolor[rgb]{0.10,0.09,0.49}{##1}}}
\expandafter\def\csname PY@tok@vg\endcsname{\def\PY@tc##1{\textcolor[rgb]{0.10,0.09,0.49}{##1}}}
\expandafter\def\csname PY@tok@vi\endcsname{\def\PY@tc##1{\textcolor[rgb]{0.10,0.09,0.49}{##1}}}
\expandafter\def\csname PY@tok@vm\endcsname{\def\PY@tc##1{\textcolor[rgb]{0.10,0.09,0.49}{##1}}}
\expandafter\def\csname PY@tok@sa\endcsname{\def\PY@tc##1{\textcolor[rgb]{0.73,0.13,0.13}{##1}}}
\expandafter\def\csname PY@tok@sb\endcsname{\def\PY@tc##1{\textcolor[rgb]{0.73,0.13,0.13}{##1}}}
\expandafter\def\csname PY@tok@sc\endcsname{\def\PY@tc##1{\textcolor[rgb]{0.73,0.13,0.13}{##1}}}
\expandafter\def\csname PY@tok@dl\endcsname{\def\PY@tc##1{\textcolor[rgb]{0.73,0.13,0.13}{##1}}}
\expandafter\def\csname PY@tok@s2\endcsname{\def\PY@tc##1{\textcolor[rgb]{0.73,0.13,0.13}{##1}}}
\expandafter\def\csname PY@tok@sh\endcsname{\def\PY@tc##1{\textcolor[rgb]{0.73,0.13,0.13}{##1}}}
\expandafter\def\csname PY@tok@s1\endcsname{\def\PY@tc##1{\textcolor[rgb]{0.73,0.13,0.13}{##1}}}
\expandafter\def\csname PY@tok@mb\endcsname{\def\PY@tc##1{\textcolor[rgb]{0.40,0.40,0.40}{##1}}}
\expandafter\def\csname PY@tok@mf\endcsname{\def\PY@tc##1{\textcolor[rgb]{0.40,0.40,0.40}{##1}}}
\expandafter\def\csname PY@tok@mh\endcsname{\def\PY@tc##1{\textcolor[rgb]{0.40,0.40,0.40}{##1}}}
\expandafter\def\csname PY@tok@mi\endcsname{\def\PY@tc##1{\textcolor[rgb]{0.40,0.40,0.40}{##1}}}
\expandafter\def\csname PY@tok@il\endcsname{\def\PY@tc##1{\textcolor[rgb]{0.40,0.40,0.40}{##1}}}
\expandafter\def\csname PY@tok@mo\endcsname{\def\PY@tc##1{\textcolor[rgb]{0.40,0.40,0.40}{##1}}}
\expandafter\def\csname PY@tok@ch\endcsname{\let\PY@it=\textit\def\PY@tc##1{\textcolor[rgb]{0.25,0.50,0.50}{##1}}}
\expandafter\def\csname PY@tok@cm\endcsname{\let\PY@it=\textit\def\PY@tc##1{\textcolor[rgb]{0.25,0.50,0.50}{##1}}}
\expandafter\def\csname PY@tok@cpf\endcsname{\let\PY@it=\textit\def\PY@tc##1{\textcolor[rgb]{0.25,0.50,0.50}{##1}}}
\expandafter\def\csname PY@tok@c1\endcsname{\let\PY@it=\textit\def\PY@tc##1{\textcolor[rgb]{0.25,0.50,0.50}{##1}}}
\expandafter\def\csname PY@tok@cs\endcsname{\let\PY@it=\textit\def\PY@tc##1{\textcolor[rgb]{0.25,0.50,0.50}{##1}}}

\def\PYZbs{\char`\\}
\def\PYZus{\char`\_}
\def\PYZob{\char`\{}
\def\PYZcb{\char`\}}
\def\PYZca{\char`\^}
\def\PYZam{\char`\&}
\def\PYZlt{\char`\<}
\def\PYZgt{\char`\>}
\def\PYZsh{\char`\#}
\def\PYZpc{\char`\%}
\def\PYZdl{\char`\$}
\def\PYZhy{\char`\-}
\def\PYZsq{\char`\'}
\def\PYZdq{\char`\"}
\def\PYZti{\char`\~}
% for compatibility with earlier versions
\def\PYZat{@}
\def\PYZlb{[}
\def\PYZrb{]}
\makeatother


    % Exact colors from NB
    \definecolor{incolor}{rgb}{0.0, 0.0, 0.5}
    \definecolor{outcolor}{rgb}{0.545, 0.0, 0.0}



    
    % Prevent overflowing lines due to hard-to-break entities
    \sloppy 
    % Setup hyperref package
    \hypersetup{
      breaklinks=true,  % so long urls are correctly broken across lines
      colorlinks=true,
      urlcolor=urlcolor,
      linkcolor=linkcolor,
      citecolor=citecolor,
      }
    % Slightly bigger margins than the latex defaults
    
    \geometry{verbose,tmargin=1in,bmargin=1in,lmargin=1in,rmargin=1in}
    
    

    \begin{document}
    
    
    \maketitle
    
    

    
    \section{Homework 5}\label{homework-5}

    \subsection{Problem 1: Faceted
Histogram}\label{problem-1-faceted-histogram}

Run the following code block to define a function which generates two
1-dimensional \texttt{numpy} arrays. The first array, called
\texttt{groups}, consists of integers between \texttt{0} and
\texttt{n\_groups\ -\ 1}, inclusive. The second array, called
\texttt{data}, consists of real numbers.

    \begin{Verbatim}[commandchars=\\\{\}]
{\color{incolor}In [{\color{incolor}1}]:} \PY{k+kn}{import} \PY{n+nn}{numpy} \PY{k}{as} \PY{n+nn}{np}
        \PY{k+kn}{from} \PY{n+nn}{matplotlib} \PY{k}{import} \PY{n}{pyplot} \PY{k}{as} \PY{n}{plt}
        
        \PY{k}{def} \PY{n+nf}{create\PYZus{}data}\PY{p}{(}\PY{n}{n}\PY{p}{,} \PY{n}{n\PYZus{}groups}\PY{p}{)}\PY{p}{:}
            \PY{l+s+sd}{\PYZdq{}\PYZdq{}\PYZdq{}}
        \PY{l+s+sd}{    generate a set of fake data with group labels. }
        \PY{l+s+sd}{    n data points and group labels are generated. }
        \PY{l+s+sd}{    n\PYZus{}groups controls the number of distinct groups. }
        \PY{l+s+sd}{    Returns an np.array() of integer group labels and an }
        \PY{l+s+sd}{    np.array() of float data. }
        \PY{l+s+sd}{    \PYZdq{}\PYZdq{}\PYZdq{}}
            
            \PY{c+c1}{\PYZsh{} random group assignments as integers between 0 and n\PYZus{}groups\PYZhy{}1, inclusive}
            \PY{n}{groups} \PY{o}{=} \PY{n}{np}\PY{o}{.}\PY{n}{random}\PY{o}{.}\PY{n}{randint}\PY{p}{(}\PY{l+m+mi}{0}\PY{p}{,} \PY{n}{n\PYZus{}groups}\PY{p}{,} \PY{n}{n}\PY{p}{)}
            
            \PY{c+c1}{\PYZsh{} function of the groups plus gaussian noise (bell curve)}
            \PY{n}{data}   \PY{o}{=} \PY{n}{np}\PY{o}{.}\PY{n}{sin}\PY{p}{(}\PY{n}{groups}\PY{p}{)} \PY{o}{+} \PY{n}{np}\PY{o}{.}\PY{n}{random}\PY{o}{.}\PY{n}{randn}\PY{p}{(}\PY{n}{n}\PY{p}{)}
            
            \PY{k}{return}\PY{p}{(}\PY{n}{groups}\PY{p}{,} \PY{n}{data}\PY{p}{)}
\end{Verbatim}


    \subsection{Part A}\label{part-a}

Write a function called \texttt{facet\_hist()}. This function should
accept five arguments:

\begin{enumerate}
\def\labelenumi{\arabic{enumi}.}
\tightlist
\item
  \texttt{groups}, the \texttt{np.array} of group labels as output by
  \texttt{create\_data()}.
\item
  \texttt{data}, the \texttt{np.array} of data as output by
  \texttt{create\_data()}.
\item
  \texttt{m\_rows}, the number of desired rows in your faceted histogram
  (explanation coming).
\item
  \texttt{m\_cols}, the number of desired columns in your faceted
  histogram (explanation coming).
\item
  \texttt{figsize}, the size of the figure.
\end{enumerate}

Your function will create faceted histograms -\/- that is, a separate
axis and histogram for each group. For example, if there are six groups
in the data, then you should be able to use the code

\begin{Shaded}
\begin{Highlighting}[]
\NormalTok{groups, data }\OperatorTok{=}\NormalTok{ create_data(}\DecValTok{1000}\NormalTok{, }\DecValTok{6}\NormalTok{)}
\NormalTok{facet_hist(groups, data, m_rows }\OperatorTok{=} \DecValTok{2}\NormalTok{, m_cols }\OperatorTok{=} \DecValTok{3}\NormalTok{, figsize }\OperatorTok{=}\NormalTok{ (}\DecValTok{6}\NormalTok{,}\DecValTok{4}\NormalTok{))}
\end{Highlighting}
\end{Shaded}

to create a plot like this:

It's fine if your group labels run left-to-right (so that the top row
has labels 0, 1, and 2 rather than 0, 2, 4).

You should also be able to change the orientation by modifying
\texttt{m\_rows}, \texttt{m\_cols}, and \texttt{figsize}.

\begin{verbatim}
facet_hist(groups, data, m_rows = 3, m_cols = 2, figsize = (4,6))
\end{verbatim}

\subsubsection{Requirements:}\label{requirements}

\begin{enumerate}
\def\labelenumi{\arabic{enumi}.}
\tightlist
\item
  Your function should work \textbf{whenever \texttt{m\_rows*m\_cols} is
  equal to the total number of groups.} Your function should first check
  that this is the case, and raise an informative \texttt{ValueError} if
  not. You may assume that there is at least one data point for each
  group label in the data supplied.
\item
  For full credit, you should not loop over the individual entries of
  \texttt{groups} or \texttt{data}. It is acceptable to loop over the
  distinct values of \texttt{groups}. In general, aim to minimize
  \texttt{for}-loops and maximize use of \texttt{Numpy} indexing.
\item
  Use of \texttt{pandas} is acceptable but unnecessary, and is unlikely
  to make your solution significantly simpler.
\item
  You should include a horizontal axis label (of your choice) along
  \textbf{only the bottom row} of axes.
\item
  You should include a vertical axis label (e.g. "Frequency") along
  \textbf{only the leftmost column of axes.}
\item
  Each axis should have an axis title of the form "Group X", as shown
  above.
\item
  Comments and docstrings!
\end{enumerate}

\subsubsection{Hints}\label{hints}

\begin{itemize}
\tightlist
\item
  If your plots look "squished," then \texttt{plt.tight\_layout()} is
  sometimes helpful. Just call it after constructing your figure, with
  no arguments.
\item
  Integer division \texttt{i\ //\ j} and remainders \texttt{i\ \%\ j}
  are helpful here, although other solutions are also possible.
\end{itemize}

    \begin{Verbatim}[commandchars=\\\{\}]
{\color{incolor}In [{\color{incolor}51}]:} \PY{c+c1}{\PYZsh{} your solution here}
         \PY{k}{def} \PY{n+nf}{facet\PYZus{}hist}\PY{p}{(}\PY{n}{groups}\PY{p}{,} \PY{n}{data}\PY{p}{,} \PY{n}{m\PYZus{}rows}\PY{p}{,} \PY{n}{m\PYZus{}cols}\PY{p}{,} \PY{n}{figsize}\PY{p}{,} \PY{o}{*}\PY{o}{*}\PY{n}{kwargs}\PY{p}{)}\PY{p}{:}
             \PY{n}{n} \PY{o}{=} \PY{n}{m\PYZus{}rows} \PY{o}{*} \PY{n}{m\PYZus{}cols}
             
             \PY{k}{if} \PY{n}{n} \PY{o}{!=} \PY{n}{groups}\PY{o}{.}\PY{n}{max}\PY{p}{(}\PY{p}{)} \PY{o}{+} \PY{l+m+mi}{1}\PY{p}{:}
                 \PY{k}{raise} \PY{n+ne}{ValueError}\PY{p}{(}\PY{l+s+s1}{\PYZsq{}}\PY{l+s+s1}{m\PYZus{}rows*m\PYZus{}cols must be equal to number of groups}\PY{l+s+s1}{\PYZsq{}}\PY{p}{)}
             
             
             \PY{n}{fig}\PY{p}{,} \PY{n}{ax} \PY{o}{=} \PY{n}{plt}\PY{o}{.}\PY{n}{subplots}\PY{p}{(}\PY{n}{m\PYZus{}rows}\PY{p}{,} \PY{n}{m\PYZus{}cols}\PY{p}{,} \PY{n}{figsize}\PY{o}{=}\PY{n}{figsize}\PY{p}{)}
             
             \PY{n}{color} \PY{o}{=} \PY{n}{kwargs}\PY{o}{.}\PY{n}{get}\PY{p}{(}\PY{l+s+s1}{\PYZsq{}}\PY{l+s+s1}{color}\PY{l+s+s1}{\PYZsq{}}\PY{p}{,} \PY{l+s+s1}{\PYZsq{}}\PY{l+s+s1}{green}\PY{l+s+s1}{\PYZsq{}}\PY{p}{)}
             \PY{n}{group\PYZus{}tracker} \PY{o}{=} \PY{l+m+mi}{0}
             \PY{k}{for} \PY{n}{row} \PY{o+ow}{in} \PY{n+nb}{range}\PY{p}{(}\PY{n}{m\PYZus{}rows}\PY{p}{)}\PY{p}{:}
                 \PY{k}{for} \PY{n}{col} \PY{o+ow}{in} \PY{n+nb}{range} \PY{p}{(}\PY{n}{m\PYZus{}cols}\PY{p}{)}\PY{p}{:}
                     \PY{n}{ix} \PY{o}{=} \PY{n}{groups} \PY{o}{==} \PY{n}{group\PYZus{}tracker}
                     \PY{n}{group\PYZus{}data} \PY{o}{=} \PY{n}{data}\PY{p}{[}\PY{n}{ix}\PY{p}{]}
                     \PY{n}{ax}\PY{p}{[}\PY{n}{row}\PY{p}{,}\PY{n}{col}\PY{p}{]}\PY{o}{.}\PY{n}{hist}\PY{p}{(}\PY{n}{group\PYZus{}data}\PY{p}{,} \PY{o}{*}\PY{o}{*}\PY{n}{kwargs}\PY{p}{)}
                     \PY{n}{ax}\PY{p}{[}\PY{n}{row}\PY{p}{,} \PY{n}{col}\PY{p}{]}\PY{o}{.}\PY{n}{set\PYZus{}title}\PY{p}{(}\PY{l+s+s1}{\PYZsq{}}\PY{l+s+s1}{Group }\PY{l+s+si}{\PYZob{}\PYZcb{}}\PY{l+s+s1}{\PYZsq{}}\PY{o}{.}\PY{n}{format}\PY{p}{(}\PY{n}{group\PYZus{}tracker}\PY{p}{)}\PY{p}{)}
                     \PY{n}{ax}\PY{p}{[}\PY{n}{row}\PY{p}{,}\PY{n}{col}\PY{p}{]}\PY{o}{.}\PY{n}{set\PYZus{}xlabel}\PY{p}{(}\PY{l+s+s1}{\PYZsq{}}\PY{l+s+s1}{Value}\PY{l+s+s1}{\PYZsq{}}\PY{p}{)}
                     \PY{n}{ax}\PY{p}{[}\PY{n}{row}\PY{p}{,}\PY{n}{col}\PY{p}{]}\PY{o}{.}\PY{n}{set\PYZus{}ylabel}\PY{p}{(}\PY{l+s+s1}{\PYZsq{}}\PY{l+s+s1}{Frequency}\PY{l+s+s1}{\PYZsq{}}\PY{p}{)}
                     \PY{n}{group\PYZus{}tracker} \PY{o}{+}\PY{o}{=} \PY{l+m+mi}{1}
             
             \PY{n}{plt}\PY{o}{.}\PY{n}{tight\PYZus{}layout}\PY{p}{(}\PY{p}{)}
\end{Verbatim}


    \begin{Verbatim}[commandchars=\\\{\}]
{\color{incolor}In [{\color{incolor}52}]:} \PY{c+c1}{\PYZsh{} test code}
         \PY{n}{groups}\PY{p}{,} \PY{n}{data} \PY{o}{=} \PY{n}{create\PYZus{}data}\PY{p}{(}\PY{l+m+mi}{1000}\PY{p}{,} \PY{l+m+mi}{6}\PY{p}{)}
         \PY{n}{facet\PYZus{}hist}\PY{p}{(}\PY{n}{groups}\PY{p}{,} \PY{n}{data}\PY{p}{,} \PY{l+m+mi}{3}\PY{p}{,} \PY{l+m+mi}{2}\PY{p}{,} \PY{n}{figsize} \PY{o}{=} \PY{p}{(}\PY{l+m+mi}{12}\PY{p}{,} \PY{l+m+mi}{8}\PY{p}{)}\PY{p}{)}
         
         \PY{n}{groups}\PY{p}{,} \PY{n}{data} \PY{o}{=} \PY{n}{create\PYZus{}data}\PY{p}{(}\PY{l+m+mi}{1000}\PY{p}{,} \PY{l+m+mi}{6}\PY{p}{)}
         \PY{n}{facet\PYZus{}hist}\PY{p}{(}\PY{n}{groups}\PY{p}{,} \PY{n}{data}\PY{p}{,} \PY{n}{m\PYZus{}rows} \PY{o}{=} \PY{l+m+mi}{2}\PY{p}{,} \PY{n}{m\PYZus{}cols} \PY{o}{=} \PY{l+m+mi}{3}\PY{p}{,} \PY{n}{figsize} \PY{o}{=} \PY{p}{(}\PY{l+m+mi}{6}\PY{p}{,}\PY{l+m+mi}{4}\PY{p}{)}\PY{p}{)}
\end{Verbatim}


    \begin{center}
    \adjustimage{max size={0.9\linewidth}{0.9\paperheight}}{output_5_0.png}
    \end{center}
    { \hspace*{\fill} \\}
    
    \begin{center}
    \adjustimage{max size={0.9\linewidth}{0.9\paperheight}}{output_5_1.png}
    \end{center}
    { \hspace*{\fill} \\}
    
    \begin{Verbatim}[commandchars=\\\{\}]
{\color{incolor}In [{\color{incolor}53}]:} \PY{c+c1}{\PYZsh{} test code}
         \PY{n}{groups}\PY{p}{,} \PY{n}{data} \PY{o}{=} \PY{n}{create\PYZus{}data}\PY{p}{(}\PY{l+m+mi}{3000}\PY{p}{,} \PY{l+m+mi}{8}\PY{p}{)}
         \PY{n}{facet\PYZus{}hist}\PY{p}{(}\PY{n}{groups}\PY{p}{,} \PY{n}{data}\PY{p}{,} \PY{l+m+mi}{4}\PY{p}{,} \PY{l+m+mi}{2}\PY{p}{,} \PY{n}{figsize} \PY{o}{=} \PY{p}{(}\PY{l+m+mi}{4}\PY{p}{,} \PY{l+m+mi}{7}\PY{p}{)}\PY{p}{)}
\end{Verbatim}


    \begin{center}
    \adjustimage{max size={0.9\linewidth}{0.9\paperheight}}{output_6_0.png}
    \end{center}
    { \hspace*{\fill} \\}
    
    \subsection{Part B}\label{part-b}

Modify your function (it's ok to modify it in place, no need for
copy/paste) so that it accepts additional \texttt{**kwargs} passed to
\texttt{ax.hist()}. For example,

\begin{verbatim}
facet_hist(groups, data, 2, 3, figsize = (6, 4), alpha = .4, color = "firebrick")
\end{verbatim}

should produce

Example output.

You should be able to run this code \textbf{without defining parameters
\texttt{alpha} and \texttt{color} for \texttt{facet\_hist()}.}

    \begin{Verbatim}[commandchars=\\\{\}]
{\color{incolor}In [{\color{incolor}58}]:} \PY{c+c1}{\PYZsh{} run this code to show that your modified function works}
         \PY{n}{groups}\PY{p}{,} \PY{n}{data} \PY{o}{=} \PY{n}{create\PYZus{}data}\PY{p}{(}\PY{l+m+mi}{1000}\PY{p}{,} \PY{l+m+mi}{6}\PY{p}{)}
         \PY{n}{facet\PYZus{}hist}\PY{p}{(}\PY{n}{groups}\PY{p}{,} \PY{n}{data}\PY{p}{,} \PY{l+m+mi}{2}\PY{p}{,} \PY{l+m+mi}{3}\PY{p}{,} \PY{n}{figsize} \PY{o}{=} \PY{p}{(}\PY{l+m+mi}{6}\PY{p}{,} \PY{l+m+mi}{4}\PY{p}{)}\PY{p}{,} \PY{n}{alpha} \PY{o}{=} \PY{o}{.}\PY{l+m+mi}{4}\PY{p}{,} \PY{n}{color} \PY{o}{=} \PY{l+s+s2}{\PYZdq{}}\PY{l+s+s2}{firebrick}\PY{l+s+s2}{\PYZdq{}}\PY{p}{)}
         
         \PY{n}{groups}\PY{p}{,} \PY{n}{data} \PY{o}{=} \PY{n}{create\PYZus{}data}\PY{p}{(}\PY{l+m+mi}{1000}\PY{p}{,} \PY{l+m+mi}{8}\PY{p}{)}
         \PY{n}{facet\PYZus{}hist}\PY{p}{(}\PY{n}{groups}\PY{p}{,} \PY{n}{data}\PY{p}{,} \PY{l+m+mi}{2}\PY{p}{,} \PY{l+m+mi}{4}\PY{p}{,} \PY{n}{figsize} \PY{o}{=} \PY{p}{(}\PY{l+m+mi}{6}\PY{p}{,} \PY{l+m+mi}{4}\PY{p}{)}\PY{p}{,} \PY{n}{alpha} \PY{o}{=} \PY{l+m+mi}{1}\PY{p}{,} \PY{n}{color} \PY{o}{=} \PY{l+s+s2}{\PYZdq{}}\PY{l+s+s2}{pink}\PY{l+s+s2}{\PYZdq{}}\PY{p}{)}
\end{Verbatim}


    \begin{center}
    \adjustimage{max size={0.9\linewidth}{0.9\paperheight}}{output_8_0.png}
    \end{center}
    { \hspace*{\fill} \\}
    
    \begin{center}
    \adjustimage{max size={0.9\linewidth}{0.9\paperheight}}{output_8_1.png}
    \end{center}
    { \hspace*{\fill} \\}
    
    \subsection{Problem 2: Scatterplot
Matrices}\label{problem-2-scatterplot-matrices}

Run the following code to download, import, and display a data set from
the 2019 World Happiness Report.

    \begin{Verbatim}[commandchars=\\\{\}]
{\color{incolor}In [{\color{incolor}181}]:} \PY{c+c1}{\PYZsh{} if you experience ConnectionRefused errors, you may instead }
          \PY{c+c1}{\PYZsh{} copy the url into your browser, save the file as data.csv }
          \PY{c+c1}{\PYZsh{} in the same directory as the notebook, and then replace the }
          \PY{c+c1}{\PYZsh{} third line with }
          \PY{c+c1}{\PYZsh{} happiness = pd.read\PYZus{}csv(\PYZdq{}data.csv\PYZdq{})}
          
          \PY{k+kn}{import} \PY{n+nn}{pandas} \PY{k}{as} \PY{n+nn}{pd}
          \PY{n}{url} \PY{o}{=} \PY{l+s+s2}{\PYZdq{}}\PY{l+s+s2}{https://philchodrow.github.io/PIC16A/datasets/world\PYZus{}happiness\PYZus{}report/2019.csv}\PY{l+s+s2}{\PYZdq{}}
          \PY{n}{happiness} \PY{o}{=} \PY{n}{pd}\PY{o}{.}\PY{n}{read\PYZus{}csv}\PY{p}{(}\PY{n}{url}\PY{p}{)}
          \PY{c+c1}{\PYZsh{} happiness}
\end{Verbatim}


    This is a \texttt{pandas} data frame. Observe the following:

\begin{enumerate}
\def\labelenumi{\arabic{enumi}.}
\tightlist
\item
  Each row corresponds to a country or region.
\item
  The \texttt{Score} column is the overall happiness score of the
  country, evaluated via surveys.
\item
  The other columns give indicators of different features of life in the
  country, including GDP, level of social support, life expectancy,
  freedom, generosity of compatriots, and perceptions of corruption in
  governmental institutions.
\end{enumerate}

You can extract each of these columns using dictionary-like syntax:

\texttt{python\ \ happiness{[}"Score"{]}}

\texttt{0\ \ \ \ \ \ 7.769\ 1\ \ \ \ \ \ 7.600\ 2\ \ \ \ \ \ 7.554\ 3\ \ \ \ \ \ 7.494\ 4\ \ \ \ \ \ 7.488\ \ \ \ \ \ \ \ ...\ \ \ 151\ \ \ \ 3.334\ 152\ \ \ \ 3.231\ 153\ \ \ \ 3.203\ 154\ \ \ \ 3.083\ 155\ \ \ \ 2.853\ Name:\ Score,\ Length:\ 156,\ dtype:\ float64}

Technically, this output is a \texttt{pandas} \texttt{Series}; however,
in this context (and most others) it's fine to simply think of it as a
1-dimensional \texttt{np.array()}.

    \subsubsection{Part A}\label{part-a}

As a warmup, create a scatterplot of the overall \texttt{Score} column
against a numerical column of your choice. Give the horizontal and
vertical axes appropriate labels. Discuss your result. Is there a
correlation? Does that correlation make sense to you?

    \begin{Verbatim}[commandchars=\\\{\}]
{\color{incolor}In [{\color{incolor}173}]:} \PY{c+c1}{\PYZsh{} plotting code here}
          \PY{n}{x} \PY{o}{=} \PY{n}{happiness}\PY{p}{[}\PY{l+s+s1}{\PYZsq{}}\PY{l+s+s1}{Social support}\PY{l+s+s1}{\PYZsq{}}\PY{p}{]}
          \PY{n}{y} \PY{o}{=} \PY{n}{happiness}\PY{p}{[}\PY{l+s+s1}{\PYZsq{}}\PY{l+s+s1}{Score}\PY{l+s+s1}{\PYZsq{}}\PY{p}{]}
          \PY{n}{plt}\PY{o}{.}\PY{n}{scatter}\PY{p}{(}\PY{n}{x}\PY{p}{,} \PY{n}{y}\PY{p}{)}
          \PY{n}{plt}\PY{o}{.}\PY{n}{xlabel}\PY{p}{(}\PY{l+s+s1}{\PYZsq{}}\PY{l+s+s1}{Social Support}\PY{l+s+s1}{\PYZsq{}}\PY{p}{)}
          \PY{n}{plt}\PY{o}{.}\PY{n}{ylabel}\PY{p}{(}\PY{l+s+s1}{\PYZsq{}}\PY{l+s+s1}{Score}\PY{l+s+s1}{\PYZsq{}}\PY{p}{)}
\end{Verbatim}


\begin{Verbatim}[commandchars=\\\{\}]
{\color{outcolor}Out[{\color{outcolor}173}]:} Text(0,0.5,'Score')
\end{Verbatim}
            
    \begin{center}
    \adjustimage{max size={0.9\linewidth}{0.9\paperheight}}{output_13_1.png}
    \end{center}
    { \hspace*{\fill} \\}
    
    \begin{center}\rule{0.5\linewidth}{\linethickness}\end{center}

\emph{Discuss here} * There seems to be a strong correlation between
social support and overall happiness score. This makes sense as people
tend to be happier when supported. -\/-\/-

    \subsection{Part B}\label{part-b}

That plot you made may have helped you understand whether or not there's
a relationship between the overall happiness score and the variable that
you chose to plot. However, there are several variables in this data
set, and we don't want to manually re-run the plot for each pair of
variables. Let's see if we can get a more systematic view of the
correlations in the data.

Write a function called \texttt{scatterplot\_matrix()}, with arguments
\texttt{cols} and \texttt{figsize}. The \texttt{cols} argument should be
a list of strings, each of which are the name of one of the columns
above, for example
\texttt{cols\ =\ {[}"Score",\ "GDP\ per\ capita",\ "Social\ support"{]}}.
Your function should create a \emph{scatterplot matrix}, like this:

\begin{Shaded}
\begin{Highlighting}[]
\NormalTok{cols }\OperatorTok{=}\NormalTok{ [}\StringTok{"Score"}\NormalTok{, }
        \StringTok{"GDP per capita"}\NormalTok{, }
        \StringTok{"Social support"}\NormalTok{]}
           
\NormalTok{scatterplot_matrix(cols,figsize }\OperatorTok{=}\NormalTok{ (}\DecValTok{7}\NormalTok{,}\DecValTok{7}\NormalTok{))}
\end{Highlighting}
\end{Shaded}

There is a separate scatterplot for each possible pair of variables. In
fact, there are two: one where the first variable is on the horizontal
axis, and one where it's on the vertical axis. Some analysts prefer to
remove half the plots to avoid redundancy, but you don't have to bother
with that. The diagonal is empty, since there's no point in
investigating the relationship between a variable and itself.

Don't forget comments and docstrings!

    \begin{Verbatim}[commandchars=\\\{\}]
{\color{incolor}In [{\color{incolor}111}]:} \PY{c+c1}{\PYZsh{} define your function}
          \PY{k}{def} \PY{n+nf}{scatterplot\PYZus{}matrix}\PY{p}{(}\PY{n}{cols}\PY{p}{,} \PY{n}{figsize}\PY{o}{=}\PY{p}{(}\PY{l+m+mi}{7}\PY{p}{,}\PY{l+m+mi}{7}\PY{p}{)}\PY{p}{)}\PY{p}{:}
              \PY{n}{fig}\PY{p}{,} \PY{n}{axs} \PY{o}{=} \PY{n}{plt}\PY{o}{.}\PY{n}{subplots}\PY{p}{(}\PY{n+nb}{len}\PY{p}{(}\PY{n}{cols}\PY{p}{)}\PY{p}{,} \PY{n+nb}{len}\PY{p}{(}\PY{n}{cols}\PY{p}{)}\PY{p}{,} \PY{n}{figsize}\PY{o}{=}\PY{n}{figsize}\PY{p}{)}
              \PY{k}{for} \PY{n}{i}\PY{p}{,} \PY{n}{col1} \PY{o+ow}{in} \PY{n+nb}{enumerate}\PY{p}{(}\PY{n}{cols}\PY{p}{)}\PY{p}{:}
                  \PY{k}{for} \PY{n}{j}\PY{p}{,} \PY{n}{col2} \PY{o+ow}{in} \PY{n+nb}{enumerate}\PY{p}{(}\PY{n}{cols}\PY{p}{)}\PY{p}{:}
                      \PY{n}{row}\PY{p}{,} \PY{n}{col} \PY{o}{=} \PY{n}{i}\PY{p}{,} \PY{n}{j}
                      \PY{n}{axs}\PY{p}{[}\PY{n}{row}\PY{p}{,}\PY{n}{col}\PY{p}{]}\PY{o}{.}\PY{n}{set\PYZus{}xlabel}\PY{p}{(}\PY{n}{col1}\PY{p}{,} \PY{n}{size}\PY{o}{=}\PY{l+m+mf}{20.0}\PY{p}{)}
                      \PY{n}{axs}\PY{p}{[}\PY{n}{row}\PY{p}{,}\PY{n}{col}\PY{p}{]}\PY{o}{.}\PY{n}{set\PYZus{}ylabel}\PY{p}{(}\PY{n}{col2}\PY{p}{,} \PY{n}{size}\PY{o}{=}\PY{l+m+mf}{20.0}\PY{p}{)}
                      \PY{k}{if} \PY{n}{col1} \PY{o}{==} \PY{n}{col2}\PY{p}{:}
                          \PY{k}{continue}
                      \PY{n}{x}\PY{p}{,} \PY{n}{y} \PY{o}{=} \PY{n}{happiness}\PY{p}{[}\PY{n}{col1}\PY{p}{]}\PY{p}{,} \PY{n}{happiness}\PY{p}{[}\PY{n}{col2}\PY{p}{]}
                      \PY{n}{corr\PYZus{}coef} \PY{o}{=} \PY{n}{np}\PY{o}{.}\PY{n}{corrcoef}\PY{p}{(}\PY{n}{x}\PY{p}{,}\PY{n}{y}\PY{p}{)}\PY{p}{[}\PY{l+m+mi}{0}\PY{p}{]}\PY{p}{[}\PY{l+m+mi}{1}\PY{p}{]}
                      \PY{n}{axs}\PY{p}{[}\PY{n}{row}\PY{p}{,} \PY{n}{col}\PY{p}{]}\PY{o}{.}\PY{n}{set\PYZus{}title}\PY{p}{(}\PY{l+s+sa}{r}\PY{l+s+s2}{\PYZdq{}}\PY{l+s+s2}{\PYZdl{}}\PY{l+s+s2}{\PYZbs{}}\PY{l+s+s2}{rho\PYZdl{} = }\PY{l+s+s2}{\PYZdq{}} \PY{o}{+} \PY{n+nb}{str}\PY{p}{(}\PY{n}{np}\PY{o}{.}\PY{n}{round}\PY{p}{(}\PY{n}{corr\PYZus{}coef}\PY{p}{,} \PY{l+m+mi}{2}\PY{p}{)}\PY{p}{)}\PY{p}{,} \PY{n}{size}\PY{o}{=}\PY{l+m+mf}{20.0}\PY{p}{)}
                      \PY{n}{axs}\PY{p}{[}\PY{n}{row}\PY{p}{,}\PY{n}{col}\PY{p}{]}\PY{o}{.}\PY{n}{scatter}\PY{p}{(}\PY{n}{x}\PY{p}{,}\PY{n}{y}\PY{p}{)}
                      
              \PY{n}{plt}\PY{o}{.}\PY{n}{tight\PYZus{}layout}\PY{p}{(}\PY{p}{)}
                      
\end{Verbatim}


    \begin{Verbatim}[commandchars=\\\{\}]
{\color{incolor}In [{\color{incolor}112}]:} \PY{c+c1}{\PYZsh{} test your code, several times if needed, and discuss the correlations you observe. }
          \PY{c+c1}{\PYZsh{} Add code cells if needed to show multiple outputs. }
          \PY{n}{scatterplot\PYZus{}matrix}\PY{p}{(} \PY{p}{[}\PY{l+s+s2}{\PYZdq{}}\PY{l+s+s2}{Score}\PY{l+s+s2}{\PYZdq{}}\PY{p}{,} 
                  \PY{l+s+s2}{\PYZdq{}}\PY{l+s+s2}{GDP per capita}\PY{l+s+s2}{\PYZdq{}}\PY{p}{,} 
                  \PY{l+s+s2}{\PYZdq{}}\PY{l+s+s2}{Social support}\PY{l+s+s2}{\PYZdq{}}\PY{p}{]}\PY{p}{,} \PY{n}{figsize}\PY{o}{=}\PY{p}{(}\PY{l+m+mi}{20}\PY{p}{,}\PY{l+m+mi}{15}\PY{p}{)}\PY{p}{)}
\end{Verbatim}


    \begin{center}
    \adjustimage{max size={0.9\linewidth}{0.9\paperheight}}{output_17_0.png}
    \end{center}
    { \hspace*{\fill} \\}
    
    \subsection{Part C}\label{part-c}

The \emph{correlation coefficient} is a measure of linear correlation
between two variables. The correlation coefficient between \(X\) and
\(Y\) is high if \(X\) tends to be high when \(Y\) is, and vice versa.
Correlation coefficients lie in the interval \([-1, 1]\).

\texttt{numpy} provides a function to conveniently compute the
correlation coefficient between two or more variables. Find it, and then
use it to add "captions" (as horizontal axis labels) to each panel of
your plot giving the correlation coefficient between the plotted
variables. For example,

\begin{Shaded}
\begin{Highlighting}[]
\NormalTok{cols }\OperatorTok{=}\NormalTok{ [}\StringTok{"Score"}\NormalTok{, }
        \StringTok{"GDP per capita"}\NormalTok{, }
        \StringTok{"Social support"}\NormalTok{]}
           
\NormalTok{scatterplot_matrix(cols,figsize }\OperatorTok{=}\NormalTok{ (}\DecValTok{7}\NormalTok{,}\DecValTok{7}\NormalTok{))}
\end{Highlighting}
\end{Shaded}

It's not required that you add the Greek letter \(\rho\) (the classical
symbol for correlation coefficients), but if you do want to, here's how.
You can also tweak the rounding as desired.

\begin{Shaded}
\begin{Highlighting}[]
\NormalTok{ax.}\BuiltInTok{set}\NormalTok{(xlabel }\OperatorTok{=} \VerbatimStringTok{r"$\textbackslash{}rho$ = "} \OperatorTok{+} \BuiltInTok{str}\NormalTok{(np.}\BuiltInTok{round}\NormalTok{(my_number, }\DecValTok{2}\NormalTok{)))}
\end{Highlighting}
\end{Shaded}

    Run your code on several different subsets of the columns. It's ok to
simply re-run your Part B results where they are and show the output
including the correlation coefficient. Discuss your findings. What
positive correlations do you observe? Do they make sense? Are there any
negative correlations? Do the quantitative results match what you see
"by eye"?

If you were going to create a model to attempt to predict overall
happiness from other indicators, which columns would you use? Why?

\begin{center}\rule{0.5\linewidth}{\linethickness}\end{center}

\emph{Discuss here} * I would use columns that have the highest
correlation with overall happiness score.

\begin{center}\rule{0.5\linewidth}{\linethickness}\end{center}

    \subsection{Problem 3: Plotting Time
Series}\label{problem-3-plotting-time-series}

Run the following code to download two time series data sets: -
Historical data on the Dow Jones Industrial Average (a composite
performance measure of the US stock market), retrieved from Yahoo
Finance. - Cumulative COVID19 cases over time, from the
\href{https://github.com/nytimes/covid-19-data}{New York Times}.

    \begin{Verbatim}[commandchars=\\\{\}]
{\color{incolor}In [{\color{incolor}113}]:} \PY{c+c1}{\PYZsh{} run this block}
          \PY{c+c1}{\PYZsh{} if you experience ConnectionRefused errors, you may instead }
          \PY{c+c1}{\PYZsh{} copy the urls into your browser, save the files as DJI.csv }
          \PY{c+c1}{\PYZsh{} and COVID.csv respectively in the same directory as the notebook.}
          \PY{c+c1}{\PYZsh{} Then, in the lines using the function pd.read\PYZus{}csv(), replace }
          \PY{c+c1}{\PYZsh{} the url with \PYZdq{}DJI.csv\PYZdq{} and \PYZdq{}COVID.csv\PYZdq{} }
          
          \PY{k+kn}{import} \PY{n+nn}{pandas} \PY{k}{as} \PY{n+nn}{pd}
          \PY{k+kn}{import} \PY{n+nn}{datetime}
          
          \PY{n}{url} \PY{o}{=} \PY{l+s+s2}{\PYZdq{}}\PY{l+s+s2}{https://query1.finance.yahoo.com/v7/finance/download/}\PY{l+s+si}{\PYZpc{}5E}\PY{l+s+s2}{DJI?period1=1580750232\PYZam{}period2=1712372632\PYZam{}interval=1d\PYZam{}events=history\PYZam{}includeAdjustedClose=true}\PY{l+s+s2}{\PYZdq{}}
          \PY{n}{DJI} \PY{o}{=} \PY{n}{pd}\PY{o}{.}\PY{n}{read\PYZus{}csv}\PY{p}{(}\PY{n}{url}\PY{p}{)}
          \PY{n}{DJI}\PY{p}{[}\PY{l+s+s1}{\PYZsq{}}\PY{l+s+s1}{date}\PY{l+s+s1}{\PYZsq{}}\PY{p}{]} \PY{o}{=} \PY{n}{pd}\PY{o}{.}\PY{n}{to\PYZus{}datetime}\PY{p}{(}\PY{n}{DJI}\PY{p}{[}\PY{l+s+s1}{\PYZsq{}}\PY{l+s+s1}{Date}\PY{l+s+s1}{\PYZsq{}}\PY{p}{]}\PY{p}{)}
          \PY{n}{DJI} \PY{o}{=} \PY{n}{DJI}\PY{o}{.}\PY{n}{drop}\PY{p}{(}\PY{p}{[}\PY{l+s+s2}{\PYZdq{}}\PY{l+s+s2}{Date}\PY{l+s+s2}{\PYZdq{}}\PY{p}{]}\PY{p}{,} \PY{n}{axis} \PY{o}{=} \PY{l+m+mi}{1}\PY{p}{)}
          
          \PY{n}{url} \PY{o}{=} \PY{l+s+s2}{\PYZdq{}}\PY{l+s+s2}{https://raw.githubusercontent.com/nytimes/covid\PYZhy{}19\PYZhy{}data/master/us.csv}\PY{l+s+s2}{\PYZdq{}}
          \PY{n}{COVID} \PY{o}{=} \PY{n}{pd}\PY{o}{.}\PY{n}{read\PYZus{}csv}\PY{p}{(}\PY{n}{url}\PY{p}{)}
          \PY{n}{COVID}\PY{p}{[}\PY{l+s+s1}{\PYZsq{}}\PY{l+s+s1}{date}\PY{l+s+s1}{\PYZsq{}}\PY{p}{]} \PY{o}{=} \PY{n}{pd}\PY{o}{.}\PY{n}{to\PYZus{}datetime}\PY{p}{(}\PY{n}{COVID}\PY{p}{[}\PY{l+s+s1}{\PYZsq{}}\PY{l+s+s1}{date}\PY{l+s+s1}{\PYZsq{}}\PY{p}{]}\PY{p}{)}
\end{Verbatim}


    \subsection{Part A}\label{part-a}

The series \texttt{COVID{[}\textquotesingle{}cases\textquotesingle{}{]}}
is essentially a \texttt{numpy} array containing the cumulative case
counts over time. The COVID19 case data is cumulative, but we would like
to see the number of new cases per day (i.e. as in
\href{https://www.google.com/search?q=covid+stats}{this kind of plot}).
Check the documentation for the \texttt{np.diff} function and figure out
what it does. Use it appropriately to construct a new array, called
\texttt{per\_day}, giving the number of new cases per day. Then, make a
new array called \texttt{per\_day\_date} that gives the appropriate date
for each case count. In particular, you will need to ensure that
\texttt{per\_day} and \texttt{per\_day\_date} have the same shape.

    \begin{Verbatim}[commandchars=\\\{\}]
{\color{incolor}In [{\color{incolor}138}]:} \PY{c+c1}{\PYZsh{} your solution here}
          \PY{n}{per\PYZus{}day} \PY{o}{=} \PY{n}{np}\PY{o}{.}\PY{n}{diff}\PY{p}{(}\PY{n}{COVID}\PY{p}{[}\PY{l+s+s1}{\PYZsq{}}\PY{l+s+s1}{cases}\PY{l+s+s1}{\PYZsq{}}\PY{p}{]}\PY{p}{)}
          \PY{n}{per\PYZus{}day\PYZus{}date} \PY{o}{=} \PY{n}{np}\PY{o}{.}\PY{n}{array}\PY{p}{(}\PY{n}{COVID}\PY{p}{[}\PY{l+s+s1}{\PYZsq{}}\PY{l+s+s1}{date}\PY{l+s+s1}{\PYZsq{}}\PY{p}{]}\PY{p}{)}\PY{p}{[}\PY{l+m+mi}{1}\PY{p}{:}\PY{p}{]}
          \PY{n}{plt}\PY{o}{.}\PY{n}{figure}\PY{p}{(}\PY{n}{figsize}\PY{o}{=}\PY{p}{(}\PY{l+m+mi}{30}\PY{p}{,}\PY{l+m+mi}{15}\PY{p}{)}\PY{p}{)}
          \PY{n}{plt}\PY{o}{.}\PY{n}{suptitle}\PY{p}{(}\PY{l+s+s1}{\PYZsq{}}\PY{l+s+s1}{Covid Cases per Day}\PY{l+s+s1}{\PYZsq{}}\PY{p}{,} \PY{n}{size}\PY{o}{=}\PY{l+m+mf}{30.0}\PY{p}{)}
          \PY{n}{plt}\PY{o}{.}\PY{n}{plot}\PY{p}{(}\PY{n}{per\PYZus{}day\PYZus{}date}\PY{p}{,}\PY{n}{per\PYZus{}day}\PY{p}{)}
          \PY{n}{plt}\PY{o}{.}\PY{n}{xlabel}\PY{p}{(}\PY{l+s+s1}{\PYZsq{}}\PY{l+s+s1}{Date}\PY{l+s+s1}{\PYZsq{}}\PY{p}{,} \PY{n}{fontsize}\PY{o}{=}\PY{l+m+mi}{20}\PY{p}{)}
          \PY{n}{plt}\PY{o}{.}\PY{n}{ylabel}\PY{p}{(}\PY{l+s+s1}{\PYZsq{}}\PY{l+s+s1}{Cases per Day}\PY{l+s+s1}{\PYZsq{}}\PY{p}{,} \PY{n}{fontsize}\PY{o}{=}\PY{l+m+mi}{20}\PY{p}{)}
          \PY{n}{plt}\PY{o}{.}\PY{n}{tick\PYZus{}params}\PY{p}{(}\PY{n}{axis}\PY{o}{=}\PY{l+s+s1}{\PYZsq{}}\PY{l+s+s1}{both}\PY{l+s+s1}{\PYZsq{}}\PY{p}{,} \PY{n}{which}\PY{o}{=}\PY{l+s+s1}{\PYZsq{}}\PY{l+s+s1}{major}\PY{l+s+s1}{\PYZsq{}}\PY{p}{,} \PY{n}{labelsize}\PY{o}{=}\PY{l+m+mi}{20}\PY{p}{)}
\end{Verbatim}


    \begin{center}
    \adjustimage{max size={0.9\linewidth}{0.9\paperheight}}{output_23_0.png}
    \end{center}
    { \hspace*{\fill} \\}
    
    \subsection{Part B}\label{part-b}

Create a figure with two very wide axes, one on top of the other (i.e.
two rows, one column). Use the \texttt{sharex} argument of
\texttt{plt.subplots()} to ensure that these two plots will share the
same horizontal axis.

Then:

\begin{enumerate}
\def\labelenumi{\arabic{enumi}.}
\tightlist
\item
  On the upper axis, plot the Dow Jones Industrial Average over time.
  For the horizontal axis use
  \texttt{DJI{[}\textquotesingle{}date\textquotesingle{}{]}}; the for
  the vertical use
  \texttt{DJI{[}\textquotesingle{}Close\textquotesingle{}{]}}.
\item
  On the lower axis, plot the variables \texttt{per\_day\_date} and
  \texttt{per\_day} to visualize the progress of the COVID19 pandemic
  over time. Use a different color for the trendline.
\end{enumerate}

Give your plot horizontal and vertical axis labels.

    \begin{Verbatim}[commandchars=\\\{\}]
{\color{incolor}In [{\color{incolor}180}]:} \PY{c+c1}{\PYZsh{} your solution here}
          \PY{c+c1}{\PYZsh{} modify this block in the remaining parts of the problem }
          \PY{k+kn}{import} \PY{n+nn}{matplotlib}
          \PY{n}{fig}\PY{p}{,} \PY{n}{axs} \PY{o}{=} \PY{n}{plt}\PY{o}{.}\PY{n}{subplots}\PY{p}{(}\PY{l+m+mi}{2}\PY{p}{,}\PY{l+m+mi}{1}\PY{p}{,} \PY{n}{sharex}\PY{o}{=}\PY{k+kc}{True}\PY{p}{,}\PY{n}{figsize}\PY{o}{=}\PY{p}{(}\PY{l+m+mi}{30}\PY{p}{,}\PY{l+m+mi}{15}\PY{p}{)}\PY{p}{)}
          \PY{n}{x1} \PY{o}{=} \PY{n}{DJI}\PY{p}{[}\PY{l+s+s1}{\PYZsq{}}\PY{l+s+s1}{date}\PY{l+s+s1}{\PYZsq{}}\PY{p}{]}
          \PY{n}{y1} \PY{o}{=} \PY{n}{DJI}\PY{p}{[}\PY{l+s+s1}{\PYZsq{}}\PY{l+s+s1}{Close}\PY{l+s+s1}{\PYZsq{}}\PY{p}{]}
          \PY{n}{axs}\PY{p}{[}\PY{l+m+mi}{0}\PY{p}{]}\PY{o}{.}\PY{n}{set}\PY{p}{(}\PY{n}{title}\PY{o}{=}\PY{l+s+s1}{\PYZsq{}}\PY{l+s+s1}{Dow Average vs Time}\PY{l+s+s1}{\PYZsq{}}\PY{p}{,} \PY{n}{ylabel}\PY{o}{=}\PY{l+s+s1}{\PYZsq{}}\PY{l+s+s1}{DJI Average}\PY{l+s+s1}{\PYZsq{}}\PY{p}{)}
          \PY{n}{axs}\PY{p}{[}\PY{l+m+mi}{0}\PY{p}{]}\PY{o}{.}\PY{n}{plot}\PY{p}{(}\PY{n}{x1}\PY{p}{,}\PY{n}{y1}\PY{p}{)}
          \PY{n}{axs}\PY{p}{[}\PY{l+m+mi}{0}\PY{p}{]}\PY{o}{.}\PY{n}{axvspan}\PY{p}{(}\PY{n}{datetime}\PY{o}{.}\PY{n}{datetime}\PY{p}{(}\PY{l+m+mi}{2020}\PY{p}{,}\PY{l+m+mi}{3}\PY{p}{,}\PY{l+m+mi}{1}\PY{p}{)}\PY{p}{,}
                        \PY{n}{datetime}\PY{o}{.}\PY{n}{datetime}\PY{p}{(}\PY{l+m+mi}{2020}\PY{p}{,}\PY{l+m+mi}{3}\PY{p}{,}\PY{l+m+mi}{30}\PY{p}{)}\PY{p}{,} 
                        \PY{n}{alpha} \PY{o}{=} \PY{o}{.}\PY{l+m+mi}{3}\PY{p}{,} 
                        \PY{n}{color} \PY{o}{=} \PY{l+s+s2}{\PYZdq{}}\PY{l+s+s2}{gray}\PY{l+s+s2}{\PYZdq{}}\PY{p}{)}
          \PY{n}{axs}\PY{p}{[}\PY{l+m+mi}{0}\PY{p}{]}\PY{o}{.}\PY{n}{text}\PY{p}{(}\PY{n}{datetime}\PY{o}{.}\PY{n}{datetime}\PY{p}{(}\PY{l+m+mi}{2020}\PY{p}{,}\PY{l+m+mi}{4}\PY{p}{,}\PY{l+m+mi}{5}\PY{p}{)}\PY{p}{,} 
                     \PY{l+m+mi}{22000}\PY{p}{,} 
                     \PY{l+s+s2}{\PYZdq{}}\PY{l+s+s2}{Covid Market Crash}\PY{l+s+s2}{\PYZdq{}}\PY{p}{,} \PY{n}{fontsize}\PY{o}{=}\PY{l+m+mi}{20}\PY{p}{)}
          \PY{n}{axs}\PY{p}{[}\PY{l+m+mi}{1}\PY{p}{]}\PY{o}{.}\PY{n}{set}\PY{p}{(}\PY{n}{title}\PY{o}{=}\PY{l+s+s1}{\PYZsq{}}\PY{l+s+s1}{Covid Cases per Day}\PY{l+s+s1}{\PYZsq{}}\PY{p}{,}\PY{n}{ylabel}\PY{o}{=}\PY{l+s+s1}{\PYZsq{}}\PY{l+s+s1}{\PYZsh{} of Cases}\PY{l+s+s1}{\PYZsq{}}\PY{p}{,} \PY{n}{xlabel}\PY{o}{=}\PY{l+s+s1}{\PYZsq{}}\PY{l+s+s1}{Date}\PY{l+s+s1}{\PYZsq{}}\PY{p}{)}
          \PY{n}{axs}\PY{p}{[}\PY{l+m+mi}{1}\PY{p}{]}\PY{o}{.}\PY{n}{plot}\PY{p}{(}\PY{n}{per\PYZus{}day\PYZus{}date}\PY{p}{,} \PY{n}{per\PYZus{}day}\PY{p}{)}
          \PY{n}{axs}\PY{p}{[}\PY{l+m+mi}{1}\PY{p}{]}\PY{o}{.}\PY{n}{axvspan}\PY{p}{(}\PY{n}{datetime}\PY{o}{.}\PY{n}{datetime}\PY{p}{(}\PY{l+m+mi}{2021}\PY{p}{,}\PY{l+m+mi}{12}\PY{p}{,}\PY{l+m+mi}{1}\PY{p}{)}\PY{p}{,}
                        \PY{n}{datetime}\PY{o}{.}\PY{n}{datetime}\PY{p}{(}\PY{l+m+mi}{2022}\PY{p}{,}\PY{l+m+mi}{2}\PY{p}{,}\PY{l+m+mi}{28}\PY{p}{)}\PY{p}{,} 
                        \PY{n}{alpha} \PY{o}{=} \PY{o}{.}\PY{l+m+mi}{3}\PY{p}{,} 
                        \PY{n}{color} \PY{o}{=} \PY{l+s+s2}{\PYZdq{}}\PY{l+s+s2}{gray}\PY{l+s+s2}{\PYZdq{}}\PY{p}{)}
          \PY{n}{axs}\PY{p}{[}\PY{l+m+mi}{1}\PY{p}{]}\PY{o}{.}\PY{n}{text}\PY{p}{(}\PY{n}{datetime}\PY{o}{.}\PY{n}{datetime}\PY{p}{(}\PY{l+m+mi}{2022}\PY{p}{,}\PY{l+m+mi}{3}\PY{p}{,}\PY{l+m+mi}{5}\PY{p}{)}\PY{p}{,} 
                     \PY{l+m+mi}{800000}\PY{p}{,} 
                     \PY{l+s+s2}{\PYZdq{}}\PY{l+s+s2}{Peak covid cases per day}\PY{l+s+s2}{\PYZdq{}}\PY{p}{,} \PY{n}{fontsize}\PY{o}{=}\PY{l+m+mi}{20}\PY{p}{)}
          \PY{n}{font} \PY{o}{=} \PY{p}{\PYZob{}}\PY{l+s+s1}{\PYZsq{}}\PY{l+s+s1}{weight}\PY{l+s+s1}{\PYZsq{}} \PY{p}{:} \PY{l+s+s1}{\PYZsq{}}\PY{l+s+s1}{bold}\PY{l+s+s1}{\PYZsq{}}\PY{p}{,}
                  \PY{l+s+s1}{\PYZsq{}}\PY{l+s+s1}{size}\PY{l+s+s1}{\PYZsq{}}   \PY{p}{:} \PY{l+m+mi}{22}\PY{p}{\PYZcb{}}
          
          \PY{n}{matplotlib}\PY{o}{.}\PY{n}{rc}\PY{p}{(}\PY{l+s+s1}{\PYZsq{}}\PY{l+s+s1}{font}\PY{l+s+s1}{\PYZsq{}}\PY{p}{,} \PY{o}{*}\PY{o}{*}\PY{n}{font}\PY{p}{)}
          \PY{n}{DJI}\PY{o}{.}\PY{n}{head}\PY{p}{(}\PY{p}{)}
\end{Verbatim}


\begin{Verbatim}[commandchars=\\\{\}]
{\color{outcolor}Out[{\color{outcolor}180}]:}            Open          High           Low         Close     Adj Close  \textbackslash{}
          0  28319.650391  28630.390625  28319.650391  28399.810547  28399.810547   
          1  28696.740234  28904.880859  28696.740234  28807.630859  28807.630859   
          2  29048.730469  29308.890625  29000.849609  29290.849609  29290.849609   
          3  29388.580078  29408.050781  29246.929688  29379.769531  29379.769531   
          4  29286.919922  29286.919922  29056.980469  29102.509766  29102.509766   
          
                Volume       date  
          0  307910000 2020-02-03  
          1  332750000 2020-02-04  
          2  357540000 2020-02-05  
          3  263700000 2020-02-06  
          4  252860000 2020-02-07  
\end{Verbatim}
            
    \begin{Verbatim}[commandchars=\\\{\}]
/home/garretg/anaconda3/lib/python3.6/site-packages/matplotlib/font\_manager.py:1328: UserWarning: findfont: Font family ['normal'] not found. Falling back to DejaVu Sans
  (prop.get\_family(), self.defaultFamily[fontext]))

    \end{Verbatim}

    \begin{center}
    \adjustimage{max size={0.9\linewidth}{0.9\paperheight}}{output_25_2.png}
    \end{center}
    { \hspace*{\fill} \\}
    
    \subsection{Part C}\label{part-c}

The command

\begin{verbatim}
ax[0].axvspan(datetime.datetime(2020,6,1),
              datetime.datetime(2020,6,30), 
              alpha = .3, 
              color = "gray")
\end{verbatim}

will add a simple rectangular shade which can be used to highlight
specific portions of a time-series. In the given code, this shade runs
through the month of June 2020. Add at least two such rectangular shades
to your figure corresponding to important time intervals. You can put
two shades on one axis, or one on each. If you're not sure what time
periods are important, just choose intervals at random. Feel free to
modify the color and transparency as desired. You can modify your figure
code from Part B -\/- no need for copy/paste.

\subsection{Part D}\label{part-d}

The command

\begin{verbatim}
ax[0].text(datetime.datetime(2020,9,15), 
           22000, 
           "penguins?\npenguins!")
\end{verbatim}

will add a fun text annotation to your plot, with the first letter in
horizontal position corresponding to September 15th, and at vertical
position 22,000. Annotate each of your shaded regions with a few words
describing their significance. Again, just modify your Part B code.

\subsection{Part E}\label{part-e}

Add an overall title, spruce up your axis labels, and add anything else
you think will make the plot look good. Again, you can just modify your
Part B code, without copy/paste.

Then, submit a job application at www.FiveThirtyEight.com and show Nate
Silver your cool data visualization.


    % Add a bibliography block to the postdoc
    
    
    
    \end{document}
