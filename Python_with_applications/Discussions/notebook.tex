
% Default to the notebook output style

    


% Inherit from the specified cell style.




    
\documentclass[11pt]{article}

    
    
    \usepackage[T1]{fontenc}
    % Nicer default font (+ math font) than Computer Modern for most use cases
    \usepackage{mathpazo}

    % Basic figure setup, for now with no caption control since it's done
    % automatically by Pandoc (which extracts ![](path) syntax from Markdown).
    \usepackage{graphicx}
    % We will generate all images so they have a width \maxwidth. This means
    % that they will get their normal width if they fit onto the page, but
    % are scaled down if they would overflow the margins.
    \makeatletter
    \def\maxwidth{\ifdim\Gin@nat@width>\linewidth\linewidth
    \else\Gin@nat@width\fi}
    \makeatother
    \let\Oldincludegraphics\includegraphics
    % Set max figure width to be 80% of text width, for now hardcoded.
    \renewcommand{\includegraphics}[1]{\Oldincludegraphics[width=.8\maxwidth]{#1}}
    % Ensure that by default, figures have no caption (until we provide a
    % proper Figure object with a Caption API and a way to capture that
    % in the conversion process - todo).
    \usepackage{caption}
    \DeclareCaptionLabelFormat{nolabel}{}
    \captionsetup{labelformat=nolabel}

    \usepackage{adjustbox} % Used to constrain images to a maximum size 
    \usepackage{xcolor} % Allow colors to be defined
    \usepackage{enumerate} % Needed for markdown enumerations to work
    \usepackage{geometry} % Used to adjust the document margins
    \usepackage{amsmath} % Equations
    \usepackage{amssymb} % Equations
    \usepackage{textcomp} % defines textquotesingle
    % Hack from http://tex.stackexchange.com/a/47451/13684:
    \AtBeginDocument{%
        \def\PYZsq{\textquotesingle}% Upright quotes in Pygmentized code
    }
    \usepackage{upquote} % Upright quotes for verbatim code
    \usepackage{eurosym} % defines \euro
    \usepackage[mathletters]{ucs} % Extended unicode (utf-8) support
    \usepackage[utf8x]{inputenc} % Allow utf-8 characters in the tex document
    \usepackage{fancyvrb} % verbatim replacement that allows latex
    \usepackage{grffile} % extends the file name processing of package graphics 
                         % to support a larger range 
    % The hyperref package gives us a pdf with properly built
    % internal navigation ('pdf bookmarks' for the table of contents,
    % internal cross-reference links, web links for URLs, etc.)
    \usepackage{hyperref}
    \usepackage{longtable} % longtable support required by pandoc >1.10
    \usepackage{booktabs}  % table support for pandoc > 1.12.2
    \usepackage[inline]{enumitem} % IRkernel/repr support (it uses the enumerate* environment)
    \usepackage[normalem]{ulem} % ulem is needed to support strikethroughs (\sout)
                                % normalem makes italics be italics, not underlines
    

    
    
    % Colors for the hyperref package
    \definecolor{urlcolor}{rgb}{0,.145,.698}
    \definecolor{linkcolor}{rgb}{.71,0.21,0.01}
    \definecolor{citecolor}{rgb}{.12,.54,.11}

    % ANSI colors
    \definecolor{ansi-black}{HTML}{3E424D}
    \definecolor{ansi-black-intense}{HTML}{282C36}
    \definecolor{ansi-red}{HTML}{E75C58}
    \definecolor{ansi-red-intense}{HTML}{B22B31}
    \definecolor{ansi-green}{HTML}{00A250}
    \definecolor{ansi-green-intense}{HTML}{007427}
    \definecolor{ansi-yellow}{HTML}{DDB62B}
    \definecolor{ansi-yellow-intense}{HTML}{B27D12}
    \definecolor{ansi-blue}{HTML}{208FFB}
    \definecolor{ansi-blue-intense}{HTML}{0065CA}
    \definecolor{ansi-magenta}{HTML}{D160C4}
    \definecolor{ansi-magenta-intense}{HTML}{A03196}
    \definecolor{ansi-cyan}{HTML}{60C6C8}
    \definecolor{ansi-cyan-intense}{HTML}{258F8F}
    \definecolor{ansi-white}{HTML}{C5C1B4}
    \definecolor{ansi-white-intense}{HTML}{A1A6B2}

    % commands and environments needed by pandoc snippets
    % extracted from the output of `pandoc -s`
    \providecommand{\tightlist}{%
      \setlength{\itemsep}{0pt}\setlength{\parskip}{0pt}}
    \DefineVerbatimEnvironment{Highlighting}{Verbatim}{commandchars=\\\{\}}
    % Add ',fontsize=\small' for more characters per line
    \newenvironment{Shaded}{}{}
    \newcommand{\KeywordTok}[1]{\textcolor[rgb]{0.00,0.44,0.13}{\textbf{{#1}}}}
    \newcommand{\DataTypeTok}[1]{\textcolor[rgb]{0.56,0.13,0.00}{{#1}}}
    \newcommand{\DecValTok}[1]{\textcolor[rgb]{0.25,0.63,0.44}{{#1}}}
    \newcommand{\BaseNTok}[1]{\textcolor[rgb]{0.25,0.63,0.44}{{#1}}}
    \newcommand{\FloatTok}[1]{\textcolor[rgb]{0.25,0.63,0.44}{{#1}}}
    \newcommand{\CharTok}[1]{\textcolor[rgb]{0.25,0.44,0.63}{{#1}}}
    \newcommand{\StringTok}[1]{\textcolor[rgb]{0.25,0.44,0.63}{{#1}}}
    \newcommand{\CommentTok}[1]{\textcolor[rgb]{0.38,0.63,0.69}{\textit{{#1}}}}
    \newcommand{\OtherTok}[1]{\textcolor[rgb]{0.00,0.44,0.13}{{#1}}}
    \newcommand{\AlertTok}[1]{\textcolor[rgb]{1.00,0.00,0.00}{\textbf{{#1}}}}
    \newcommand{\FunctionTok}[1]{\textcolor[rgb]{0.02,0.16,0.49}{{#1}}}
    \newcommand{\RegionMarkerTok}[1]{{#1}}
    \newcommand{\ErrorTok}[1]{\textcolor[rgb]{1.00,0.00,0.00}{\textbf{{#1}}}}
    \newcommand{\NormalTok}[1]{{#1}}
    
    % Additional commands for more recent versions of Pandoc
    \newcommand{\ConstantTok}[1]{\textcolor[rgb]{0.53,0.00,0.00}{{#1}}}
    \newcommand{\SpecialCharTok}[1]{\textcolor[rgb]{0.25,0.44,0.63}{{#1}}}
    \newcommand{\VerbatimStringTok}[1]{\textcolor[rgb]{0.25,0.44,0.63}{{#1}}}
    \newcommand{\SpecialStringTok}[1]{\textcolor[rgb]{0.73,0.40,0.53}{{#1}}}
    \newcommand{\ImportTok}[1]{{#1}}
    \newcommand{\DocumentationTok}[1]{\textcolor[rgb]{0.73,0.13,0.13}{\textit{{#1}}}}
    \newcommand{\AnnotationTok}[1]{\textcolor[rgb]{0.38,0.63,0.69}{\textbf{\textit{{#1}}}}}
    \newcommand{\CommentVarTok}[1]{\textcolor[rgb]{0.38,0.63,0.69}{\textbf{\textit{{#1}}}}}
    \newcommand{\VariableTok}[1]{\textcolor[rgb]{0.10,0.09,0.49}{{#1}}}
    \newcommand{\ControlFlowTok}[1]{\textcolor[rgb]{0.00,0.44,0.13}{\textbf{{#1}}}}
    \newcommand{\OperatorTok}[1]{\textcolor[rgb]{0.40,0.40,0.40}{{#1}}}
    \newcommand{\BuiltInTok}[1]{{#1}}
    \newcommand{\ExtensionTok}[1]{{#1}}
    \newcommand{\PreprocessorTok}[1]{\textcolor[rgb]{0.74,0.48,0.00}{{#1}}}
    \newcommand{\AttributeTok}[1]{\textcolor[rgb]{0.49,0.56,0.16}{{#1}}}
    \newcommand{\InformationTok}[1]{\textcolor[rgb]{0.38,0.63,0.69}{\textbf{\textit{{#1}}}}}
    \newcommand{\WarningTok}[1]{\textcolor[rgb]{0.38,0.63,0.69}{\textbf{\textit{{#1}}}}}
    
    
    % Define a nice break command that doesn't care if a line doesn't already
    % exist.
    \def\br{\hspace*{\fill} \\* }
    % Math Jax compatability definitions
    \def\gt{>}
    \def\lt{<}
    % Document parameters
    \title{hello}
    
    
    

    % Pygments definitions
    
\makeatletter
\def\PY@reset{\let\PY@it=\relax \let\PY@bf=\relax%
    \let\PY@ul=\relax \let\PY@tc=\relax%
    \let\PY@bc=\relax \let\PY@ff=\relax}
\def\PY@tok#1{\csname PY@tok@#1\endcsname}
\def\PY@toks#1+{\ifx\relax#1\empty\else%
    \PY@tok{#1}\expandafter\PY@toks\fi}
\def\PY@do#1{\PY@bc{\PY@tc{\PY@ul{%
    \PY@it{\PY@bf{\PY@ff{#1}}}}}}}
\def\PY#1#2{\PY@reset\PY@toks#1+\relax+\PY@do{#2}}

\expandafter\def\csname PY@tok@w\endcsname{\def\PY@tc##1{\textcolor[rgb]{0.73,0.73,0.73}{##1}}}
\expandafter\def\csname PY@tok@c\endcsname{\let\PY@it=\textit\def\PY@tc##1{\textcolor[rgb]{0.25,0.50,0.50}{##1}}}
\expandafter\def\csname PY@tok@cp\endcsname{\def\PY@tc##1{\textcolor[rgb]{0.74,0.48,0.00}{##1}}}
\expandafter\def\csname PY@tok@k\endcsname{\let\PY@bf=\textbf\def\PY@tc##1{\textcolor[rgb]{0.00,0.50,0.00}{##1}}}
\expandafter\def\csname PY@tok@kp\endcsname{\def\PY@tc##1{\textcolor[rgb]{0.00,0.50,0.00}{##1}}}
\expandafter\def\csname PY@tok@kt\endcsname{\def\PY@tc##1{\textcolor[rgb]{0.69,0.00,0.25}{##1}}}
\expandafter\def\csname PY@tok@o\endcsname{\def\PY@tc##1{\textcolor[rgb]{0.40,0.40,0.40}{##1}}}
\expandafter\def\csname PY@tok@ow\endcsname{\let\PY@bf=\textbf\def\PY@tc##1{\textcolor[rgb]{0.67,0.13,1.00}{##1}}}
\expandafter\def\csname PY@tok@nb\endcsname{\def\PY@tc##1{\textcolor[rgb]{0.00,0.50,0.00}{##1}}}
\expandafter\def\csname PY@tok@nf\endcsname{\def\PY@tc##1{\textcolor[rgb]{0.00,0.00,1.00}{##1}}}
\expandafter\def\csname PY@tok@nc\endcsname{\let\PY@bf=\textbf\def\PY@tc##1{\textcolor[rgb]{0.00,0.00,1.00}{##1}}}
\expandafter\def\csname PY@tok@nn\endcsname{\let\PY@bf=\textbf\def\PY@tc##1{\textcolor[rgb]{0.00,0.00,1.00}{##1}}}
\expandafter\def\csname PY@tok@ne\endcsname{\let\PY@bf=\textbf\def\PY@tc##1{\textcolor[rgb]{0.82,0.25,0.23}{##1}}}
\expandafter\def\csname PY@tok@nv\endcsname{\def\PY@tc##1{\textcolor[rgb]{0.10,0.09,0.49}{##1}}}
\expandafter\def\csname PY@tok@no\endcsname{\def\PY@tc##1{\textcolor[rgb]{0.53,0.00,0.00}{##1}}}
\expandafter\def\csname PY@tok@nl\endcsname{\def\PY@tc##1{\textcolor[rgb]{0.63,0.63,0.00}{##1}}}
\expandafter\def\csname PY@tok@ni\endcsname{\let\PY@bf=\textbf\def\PY@tc##1{\textcolor[rgb]{0.60,0.60,0.60}{##1}}}
\expandafter\def\csname PY@tok@na\endcsname{\def\PY@tc##1{\textcolor[rgb]{0.49,0.56,0.16}{##1}}}
\expandafter\def\csname PY@tok@nt\endcsname{\let\PY@bf=\textbf\def\PY@tc##1{\textcolor[rgb]{0.00,0.50,0.00}{##1}}}
\expandafter\def\csname PY@tok@nd\endcsname{\def\PY@tc##1{\textcolor[rgb]{0.67,0.13,1.00}{##1}}}
\expandafter\def\csname PY@tok@s\endcsname{\def\PY@tc##1{\textcolor[rgb]{0.73,0.13,0.13}{##1}}}
\expandafter\def\csname PY@tok@sd\endcsname{\let\PY@it=\textit\def\PY@tc##1{\textcolor[rgb]{0.73,0.13,0.13}{##1}}}
\expandafter\def\csname PY@tok@si\endcsname{\let\PY@bf=\textbf\def\PY@tc##1{\textcolor[rgb]{0.73,0.40,0.53}{##1}}}
\expandafter\def\csname PY@tok@se\endcsname{\let\PY@bf=\textbf\def\PY@tc##1{\textcolor[rgb]{0.73,0.40,0.13}{##1}}}
\expandafter\def\csname PY@tok@sr\endcsname{\def\PY@tc##1{\textcolor[rgb]{0.73,0.40,0.53}{##1}}}
\expandafter\def\csname PY@tok@ss\endcsname{\def\PY@tc##1{\textcolor[rgb]{0.10,0.09,0.49}{##1}}}
\expandafter\def\csname PY@tok@sx\endcsname{\def\PY@tc##1{\textcolor[rgb]{0.00,0.50,0.00}{##1}}}
\expandafter\def\csname PY@tok@m\endcsname{\def\PY@tc##1{\textcolor[rgb]{0.40,0.40,0.40}{##1}}}
\expandafter\def\csname PY@tok@gh\endcsname{\let\PY@bf=\textbf\def\PY@tc##1{\textcolor[rgb]{0.00,0.00,0.50}{##1}}}
\expandafter\def\csname PY@tok@gu\endcsname{\let\PY@bf=\textbf\def\PY@tc##1{\textcolor[rgb]{0.50,0.00,0.50}{##1}}}
\expandafter\def\csname PY@tok@gd\endcsname{\def\PY@tc##1{\textcolor[rgb]{0.63,0.00,0.00}{##1}}}
\expandafter\def\csname PY@tok@gi\endcsname{\def\PY@tc##1{\textcolor[rgb]{0.00,0.63,0.00}{##1}}}
\expandafter\def\csname PY@tok@gr\endcsname{\def\PY@tc##1{\textcolor[rgb]{1.00,0.00,0.00}{##1}}}
\expandafter\def\csname PY@tok@ge\endcsname{\let\PY@it=\textit}
\expandafter\def\csname PY@tok@gs\endcsname{\let\PY@bf=\textbf}
\expandafter\def\csname PY@tok@gp\endcsname{\let\PY@bf=\textbf\def\PY@tc##1{\textcolor[rgb]{0.00,0.00,0.50}{##1}}}
\expandafter\def\csname PY@tok@go\endcsname{\def\PY@tc##1{\textcolor[rgb]{0.53,0.53,0.53}{##1}}}
\expandafter\def\csname PY@tok@gt\endcsname{\def\PY@tc##1{\textcolor[rgb]{0.00,0.27,0.87}{##1}}}
\expandafter\def\csname PY@tok@err\endcsname{\def\PY@bc##1{\setlength{\fboxsep}{0pt}\fcolorbox[rgb]{1.00,0.00,0.00}{1,1,1}{\strut ##1}}}
\expandafter\def\csname PY@tok@kc\endcsname{\let\PY@bf=\textbf\def\PY@tc##1{\textcolor[rgb]{0.00,0.50,0.00}{##1}}}
\expandafter\def\csname PY@tok@kd\endcsname{\let\PY@bf=\textbf\def\PY@tc##1{\textcolor[rgb]{0.00,0.50,0.00}{##1}}}
\expandafter\def\csname PY@tok@kn\endcsname{\let\PY@bf=\textbf\def\PY@tc##1{\textcolor[rgb]{0.00,0.50,0.00}{##1}}}
\expandafter\def\csname PY@tok@kr\endcsname{\let\PY@bf=\textbf\def\PY@tc##1{\textcolor[rgb]{0.00,0.50,0.00}{##1}}}
\expandafter\def\csname PY@tok@bp\endcsname{\def\PY@tc##1{\textcolor[rgb]{0.00,0.50,0.00}{##1}}}
\expandafter\def\csname PY@tok@fm\endcsname{\def\PY@tc##1{\textcolor[rgb]{0.00,0.00,1.00}{##1}}}
\expandafter\def\csname PY@tok@vc\endcsname{\def\PY@tc##1{\textcolor[rgb]{0.10,0.09,0.49}{##1}}}
\expandafter\def\csname PY@tok@vg\endcsname{\def\PY@tc##1{\textcolor[rgb]{0.10,0.09,0.49}{##1}}}
\expandafter\def\csname PY@tok@vi\endcsname{\def\PY@tc##1{\textcolor[rgb]{0.10,0.09,0.49}{##1}}}
\expandafter\def\csname PY@tok@vm\endcsname{\def\PY@tc##1{\textcolor[rgb]{0.10,0.09,0.49}{##1}}}
\expandafter\def\csname PY@tok@sa\endcsname{\def\PY@tc##1{\textcolor[rgb]{0.73,0.13,0.13}{##1}}}
\expandafter\def\csname PY@tok@sb\endcsname{\def\PY@tc##1{\textcolor[rgb]{0.73,0.13,0.13}{##1}}}
\expandafter\def\csname PY@tok@sc\endcsname{\def\PY@tc##1{\textcolor[rgb]{0.73,0.13,0.13}{##1}}}
\expandafter\def\csname PY@tok@dl\endcsname{\def\PY@tc##1{\textcolor[rgb]{0.73,0.13,0.13}{##1}}}
\expandafter\def\csname PY@tok@s2\endcsname{\def\PY@tc##1{\textcolor[rgb]{0.73,0.13,0.13}{##1}}}
\expandafter\def\csname PY@tok@sh\endcsname{\def\PY@tc##1{\textcolor[rgb]{0.73,0.13,0.13}{##1}}}
\expandafter\def\csname PY@tok@s1\endcsname{\def\PY@tc##1{\textcolor[rgb]{0.73,0.13,0.13}{##1}}}
\expandafter\def\csname PY@tok@mb\endcsname{\def\PY@tc##1{\textcolor[rgb]{0.40,0.40,0.40}{##1}}}
\expandafter\def\csname PY@tok@mf\endcsname{\def\PY@tc##1{\textcolor[rgb]{0.40,0.40,0.40}{##1}}}
\expandafter\def\csname PY@tok@mh\endcsname{\def\PY@tc##1{\textcolor[rgb]{0.40,0.40,0.40}{##1}}}
\expandafter\def\csname PY@tok@mi\endcsname{\def\PY@tc##1{\textcolor[rgb]{0.40,0.40,0.40}{##1}}}
\expandafter\def\csname PY@tok@il\endcsname{\def\PY@tc##1{\textcolor[rgb]{0.40,0.40,0.40}{##1}}}
\expandafter\def\csname PY@tok@mo\endcsname{\def\PY@tc##1{\textcolor[rgb]{0.40,0.40,0.40}{##1}}}
\expandafter\def\csname PY@tok@ch\endcsname{\let\PY@it=\textit\def\PY@tc##1{\textcolor[rgb]{0.25,0.50,0.50}{##1}}}
\expandafter\def\csname PY@tok@cm\endcsname{\let\PY@it=\textit\def\PY@tc##1{\textcolor[rgb]{0.25,0.50,0.50}{##1}}}
\expandafter\def\csname PY@tok@cpf\endcsname{\let\PY@it=\textit\def\PY@tc##1{\textcolor[rgb]{0.25,0.50,0.50}{##1}}}
\expandafter\def\csname PY@tok@c1\endcsname{\let\PY@it=\textit\def\PY@tc##1{\textcolor[rgb]{0.25,0.50,0.50}{##1}}}
\expandafter\def\csname PY@tok@cs\endcsname{\let\PY@it=\textit\def\PY@tc##1{\textcolor[rgb]{0.25,0.50,0.50}{##1}}}

\def\PYZbs{\char`\\}
\def\PYZus{\char`\_}
\def\PYZob{\char`\{}
\def\PYZcb{\char`\}}
\def\PYZca{\char`\^}
\def\PYZam{\char`\&}
\def\PYZlt{\char`\<}
\def\PYZgt{\char`\>}
\def\PYZsh{\char`\#}
\def\PYZpc{\char`\%}
\def\PYZdl{\char`\$}
\def\PYZhy{\char`\-}
\def\PYZsq{\char`\'}
\def\PYZdq{\char`\"}
\def\PYZti{\char`\~}
% for compatibility with earlier versions
\def\PYZat{@}
\def\PYZlb{[}
\def\PYZrb{]}
\makeatother


    % Exact colors from NB
    \definecolor{incolor}{rgb}{0.0, 0.0, 0.5}
    \definecolor{outcolor}{rgb}{0.545, 0.0, 0.0}



    
    % Prevent overflowing lines due to hard-to-break entities
    \sloppy 
    % Setup hyperref package
    \hypersetup{
      breaklinks=true,  % so long urls are correctly broken across lines
      colorlinks=true,
      urlcolor=urlcolor,
      linkcolor=linkcolor,
      citecolor=citecolor,
      }
    % Slightly bigger margins than the latex defaults
    
    \geometry{verbose,tmargin=1in,bmargin=1in,lmargin=1in,rmargin=1in}
    
    

    \begin{document}
    
    
    \maketitle
    
    

    
    \section{Hello, PIC16A}\label{hello-pic16a}

\subsubsection{Group Members and Roles}\label{group-members-and-roles}

\begin{itemize}
\tightlist
\item
  Group Member 1 (Role)
\item
  Group Member 2 (Role)
\item
  Group Member 3 (Role)
\end{itemize}

\subsubsection{Recap of Roles}\label{recap-of-roles}

Today, the group member whose first name comes first in the alphabet is
the \textbf{Driver}. They are responsible for opening a copy of this
worksheet and sharing their screen. They are the only one who needs to
modify the worksheet directly.

The group member whose first name comes next in the alphabet is the
\textbf{Proposer}. They are responsible for suggesting ideas for what
the Driver should do next.

The final group member is the \textbf{Reviewer}. They are responsible
for finding improvements, giving feedback, and researching questions if
the group gets stuck.

You'll be rotating these roles throughout the quarter.

\textbf{Add your names and roles beside the bullet points above.}

\subsubsection{Cameras and Microphones}\label{cameras-and-microphones}

\textbf{Please turn your cameras on.} If you are not comfortable sharing
your surroundings, feel free to use a virtual background. It's
completely fine to be in pajamas, in bed, eating, drinking tea, etc.
during Discussion, as long as you're ready to focus and contribute to
the group.

\textbf{Please turn your microphones on.} Unless there is a disruptive
noise in your environment, you should never be muted during Discussion.

\section{§1. Hello Team (\textasciitilde{}15
minutes)}\label{hello-team-15-minutes}

There's plenty of time for this part of the activity, so please feel
free to ask each other additional questions.

Starting with the Reviewer, introduce yourselves! First, share with each
other the following.

\begin{enumerate}
\def\labelenumi{\arabic{enumi}.}
\tightlist
\item
  Your name.
\item
  Your area or planned area of study at UCLA.
\item
  A fun fact about you.
\end{enumerate}

Once each of you have spoken, move on to the next question:

\begin{quote}
\emph{Why are you interested in Python programming?}
\end{quote}

Once each of you have shared your interest in Python programming move on
to the following question. You might want to spend a minute or so
thinking about this one.

\begin{quote}
\emph{If you were a vegetable, which vegetable would you be, \textbf{and
why}?}
\end{quote}

Make sure that you have both selected a vegetable and a reason. Note
that this question is \textbf{not} the same as "what is your favorite
vegetable?"

For the purposes of this question, botanical vegetables and culinary
vegetables are valid choices. For example, "tomato" or "squash" is fully
acceptable, as long as you have a reason.

    \section{§2. Hello Jupyter (\textasciitilde{}15
minutes)}\label{hello-jupyter-15-minutes}

The most important thing you will need to do in Jupyter is read, write,
and run Python code. Write the following code, exactly as it appears, in
the \emph{code cell} below, underneath the \texttt{\#\ comment}.

\begin{Shaded}
\begin{Highlighting}[]
\BuiltInTok{print}\NormalTok{(}\StringTok{"to boldly go where no one has gone before"}\NormalTok{)}
\end{Highlighting}
\end{Shaded}

Then, with your cursor still in the code cell, hit
\texttt{shift\ +\ enter} on your keyboard to run the code. You may also
see a "Play" or "Run" button in the Jupyter toolbar which you can use to
achieve the same result and advance to the next cell. There is also a
keyboard shortcut to run the code and not advance to the next cell,
which is often convenient. The shortcut depends on your operating system
and whether you are using Jupyter Notebook or Jupyter Lab. These
shortcuts are shown in the menu bar.

    \begin{Verbatim}[commandchars=\\\{\}]
{\color{incolor}In [{\color{incolor}1}]:} \PY{c+c1}{\PYZsh{} write your code here}
        \PY{n+nb}{print}\PY{p}{(}\PY{l+s+s2}{\PYZdq{}}\PY{l+s+s2}{to boldly go where no one has gone before}\PY{l+s+s2}{\PYZdq{}}\PY{p}{)}
\end{Verbatim}


    \begin{Verbatim}[commandchars=\\\{\}]
to boldly go where no one has gone before

    \end{Verbatim}

    An important feature of the Jupyter interface is that the Python
environment is persistent between code cells. That means that variables
that you define in one code cell can be used in later code cells.

In the cell below, type:

\begin{Shaded}
\begin{Highlighting}[]
\NormalTok{vegetables }\OperatorTok{=}\NormalTok{ \{}\StringTok{"driver"}\NormalTok{ : }\StringTok{"radish"}\NormalTok{, }\StringTok{"proposer"}\NormalTok{ : }\StringTok{"carrot"}\NormalTok{, }\StringTok{"reviewer"}\NormalTok{ : }\StringTok{"onion"}\NormalTok{\}}
\end{Highlighting}
\end{Shaded}

Replace the vegetables with the vegetables you chose in §1.

Run the code cell (\texttt{shift\ +\ enter}).

    \begin{Verbatim}[commandchars=\\\{\}]
{\color{incolor}In [{\color{incolor}3}]:} \PY{c+c1}{\PYZsh{} write your code here}
        \PY{n}{vegetables} \PY{o}{=} \PY{p}{\PYZob{}}\PY{l+s+s2}{\PYZdq{}}\PY{l+s+s2}{driver}\PY{l+s+s2}{\PYZdq{}} \PY{p}{:} \PY{l+s+s2}{\PYZdq{}}\PY{l+s+s2}{radish}\PY{l+s+s2}{\PYZdq{}}\PY{p}{,} \PY{l+s+s2}{\PYZdq{}}\PY{l+s+s2}{proposer}\PY{l+s+s2}{\PYZdq{}} \PY{p}{:} \PY{l+s+s2}{\PYZdq{}}\PY{l+s+s2}{carrot}\PY{l+s+s2}{\PYZdq{}}\PY{p}{,} \PY{l+s+s2}{\PYZdq{}}\PY{l+s+s2}{reviewer}\PY{l+s+s2}{\PYZdq{}} \PY{p}{:} \PY{l+s+s2}{\PYZdq{}}\PY{l+s+s2}{onion}\PY{l+s+s2}{\PYZdq{}}\PY{p}{\PYZcb{}}
\end{Verbatim}


    The variable \texttt{vegetables} is now part of the Python session and
is available for future use. For example, type the following code in the
cell below and run it.

\begin{Shaded}
\begin{Highlighting}[]
\BuiltInTok{print}\NormalTok{(}\StringTok{"If they were a vegetable, the Driver would be a "} \OperatorTok{+}\NormalTok{ vegetables[}\StringTok{"driver"}\NormalTok{] }\OperatorTok{+} \StringTok{"."}\NormalTok{)}
\end{Highlighting}
\end{Shaded}

    \begin{Verbatim}[commandchars=\\\{\}]
{\color{incolor}In [{\color{incolor}4}]:} \PY{c+c1}{\PYZsh{} write your code here}
        \PY{n+nb}{print}\PY{p}{(}\PY{l+s+s2}{\PYZdq{}}\PY{l+s+s2}{If they were a vegetable, the Driver would be a }\PY{l+s+s2}{\PYZdq{}} \PY{o}{+} \PY{n}{vegetables}\PY{p}{[}\PY{l+s+s2}{\PYZdq{}}\PY{l+s+s2}{driver}\PY{l+s+s2}{\PYZdq{}}\PY{p}{]} \PY{o}{+} \PY{l+s+s2}{\PYZdq{}}\PY{l+s+s2}{.}\PY{l+s+s2}{\PYZdq{}}\PY{p}{)}
\end{Verbatim}


    \begin{Verbatim}[commandchars=\\\{\}]
If they were a vegetable, the Driver would be a radish.

    \end{Verbatim}

    Sometimes, you'll need to write some code, but there won't be a code
cell present. In this case, you can use the \texttt{+} symbol on the
toolbar to add a new code cell. Add a new cell now, immediately below
this text, and type the following code:

\begin{Shaded}
\begin{Highlighting}[]
\BuiltInTok{print}\NormalTok{(}\StringTok{"If they were a vegetable, the Reviewer would be a "} \OperatorTok{+}\NormalTok{ vegetables[}\StringTok{"reviewer"}\NormalTok{] }\OperatorTok{+} \StringTok{"."}\NormalTok{)}
\end{Highlighting}
\end{Shaded}

    \begin{Verbatim}[commandchars=\\\{\}]
{\color{incolor}In [{\color{incolor}5}]:} \PY{n+nb}{print}\PY{p}{(}\PY{l+s+s2}{\PYZdq{}}\PY{l+s+s2}{If they were a vegetable, the Reviewer would be a }\PY{l+s+s2}{\PYZdq{}} \PY{o}{+} \PY{n}{vegetables}\PY{p}{[}\PY{l+s+s2}{\PYZdq{}}\PY{l+s+s2}{reviewer}\PY{l+s+s2}{\PYZdq{}}\PY{p}{]} \PY{o}{+} \PY{l+s+s2}{\PYZdq{}}\PY{l+s+s2}{.}\PY{l+s+s2}{\PYZdq{}}\PY{p}{)}
\end{Verbatim}


    \begin{Verbatim}[commandchars=\\\{\}]
If they were a vegetable, the Reviewer would be a onion.

    \end{Verbatim}

    On occasion, you may find it useful to \emph{restart the kernel}.
Restarting the kernel has the effect of erasing all variables in the
current Python session.

\textbf{Restart the kernel by clicking the ↻ button in the Jupyter
toolbar.} If asked whether you are sure you want to restart the kernel,
choose "yes"

Create a code cell below, and type in the following code:

\begin{Shaded}
\begin{Highlighting}[]
\BuiltInTok{print}\NormalTok{(}\StringTok{"If they were a vegetable, the Proposer would be a "} \OperatorTok{+}\NormalTok{ vegetables[}\StringTok{"proposer"}\NormalTok{] }\OperatorTok{+} \StringTok{"."}\NormalTok{)}
\end{Highlighting}
\end{Shaded}

    \begin{Verbatim}[commandchars=\\\{\}]
{\color{incolor}In [{\color{incolor}6}]:} \PY{n+nb}{print}\PY{p}{(}\PY{l+s+s2}{\PYZdq{}}\PY{l+s+s2}{If they were a vegetable, the Proposer would be a }\PY{l+s+s2}{\PYZdq{}} \PY{o}{+} \PY{n}{vegetables}\PY{p}{[}\PY{l+s+s2}{\PYZdq{}}\PY{l+s+s2}{proposer}\PY{l+s+s2}{\PYZdq{}}\PY{p}{]} \PY{o}{+} \PY{l+s+s2}{\PYZdq{}}\PY{l+s+s2}{.}\PY{l+s+s2}{\PYZdq{}}\PY{p}{)}
\end{Verbatim}


    \begin{Verbatim}[commandchars=\\\{\}]
If they were a vegetable, the Proposer would be a carrot.

    \end{Verbatim}

    You should observe a
\texttt{NameError:\ name\ \textquotesingle{}vegetables\textquotesingle{}\ is\ not\ defined}.
This is because, after restarting the kernel, the variable
\texttt{vegetables} is no longer present in our namespace. To get it
back, go back to the code cell in which you first defined this variable
and run it. Then, the cell above will run without error.

The toolbar also contains useful tools for duplicating cells, removing
cells, and moving cells up and down.

    \section{§3. Hello Markdown (\textasciitilde{}10
minutes)}\label{hello-markdown-10-minutes}

Throughout PIC16A, you will frequently be asked to write explanatory
text in which you describe your problem-solving strategy and the
structure of your code. One way to do this will be in comments and
docstrings, which we will introduce later. Another important way is with
\emph{Markdown cells.}
\href{https://en.wikipedia.org/wiki/Markdown\#Example}{Markdown} is a
simple, readable way to create formatted text. It was invented by John
Gruber and Aaron Swartz. This
\href{https://commonmark.org/help/}{cheatsheet} contains a useful and
concise set of instructions for Markdown editing.

To create a Markdown cell, first create a code cell like you did before.
Then, with your cursor in the code cell, click the dropdown menu in the
toolbar that says "Code". Select "Markdown" from this menu instead.

Return to the cell and type the following characters, exactly as they
appear:

\texttt{*To\ boldly\ go*\ **where\ no\ one\ has\ gone\ before.**}

Then, "run" the Markdown cell (\texttt{shift\ +\ enter} again) to see
your text rendered with italic and bold formatting.

{[}add markdown cell here{]}

    \emph{To boldly go} \textbf{where no one has gone before.}

    Now, work with your group to reproduce in Markdown the following
algorithm for enjoying a relaxing cup of tea.

{[}add markdown cell here{]}

    \section{\texorpdfstring{\textbf{Making
Tea}}{Making Tea}}\label{making-tea}

\begin{enumerate}
\def\labelenumi{\arabic{enumi}.}
\tightlist
\item
  Boil water.
\item
  Place tea bag in mug.
\item
  Pour water in mug. \textbf{Be careful, water is hot!}
\item
  \emph{Optional: add milk and/or surgar.}
\item
  Drink tea.

  \begin{itemize}
  \tightlist
  \item
    \texttt{while\ cup\ is\ not\ empty}

    \begin{itemize}
    \tightlist
    \item
      Carefully raise mug to mouth.
    \item
      Take a sip.
    \end{itemize}
  \end{itemize}
\item
  Clean mug.
\end{enumerate}

    \section{§4. Goodbye Worksheet (\textasciitilde{}5
minutes)}\label{goodbye-worksheet-5-minutes}

Great job! It's time to submit the worksheet. To do so, the Driver
should \textbf{convert the worksheet to PDF} and then submit the PDF.
There are a few valid ways to do this. \texttt{File} -\textgreater{}
\texttt{Download\ As} on the menu.

\begin{enumerate}
\def\labelenumi{\arabic{enumi}.}
\tightlist
\item
  It is likely that the \(\LaTeX\) markup language was installed with
  Anaconda. \(\LaTeX\) is a popular way to create scientific reports and
  presentations. In this case, you can download the worksheet as a PDF
  by choosing
  \texttt{File\ -\textgreater{}\ Download\ As\ -\textgreater{}\ PDF\ via\ LaTeX\ (.pdf)}
  on the Jupyter toolbar. This will produce an attractive report.
\item
  If you would prefer not to use \(\LaTeX\), you can also choose
  \texttt{File\ -\textgreater{}\ Print\ Preview}. This will open a new
  window with a simplified view of the notebook suitable for printing.
  Choose the \texttt{Save\ as\ PDF} option from your print menu.
\end{enumerate}

After obtaining a PDF, submit it on CCLE. Only one member of the group
(the Driver) needs to submit the worksheet.

\textbf{Note on grading}: Discussion worksheets are graded on
\emph{participation}. This means that it is not necessary to complete
the entire worksheet in order to receive full credit, provided that you
are actively working on the activity for the entirety of the Discussion
period. If, however, you would like to meet up later with your partners
to finish up the activity, then please do -\/- this will be very helpful
for your learning (and therefore your performance on exams).

    \begin{Verbatim}[commandchars=\\\{\}]
{\color{incolor}In [{\color{incolor}8}]:} \PY{l+m+mi}{2}\PY{o}{+}\PY{l+m+mi}{2}
\end{Verbatim}


\begin{Verbatim}[commandchars=\\\{\}]
{\color{outcolor}Out[{\color{outcolor}8}]:} 4
\end{Verbatim}
            
    \begin{Verbatim}[commandchars=\\\{\}]
{\color{incolor}In [{\color{incolor}9}]:} \PY{n}{x} \PY{o}{=} \PY{l+m+mi}{5}
\end{Verbatim}


    \begin{Verbatim}[commandchars=\\\{\}]
{\color{incolor}In [{\color{incolor}10}]:} \PY{l+m+mi}{2}\PY{o}{*}\PY{n}{x}
\end{Verbatim}


\begin{Verbatim}[commandchars=\\\{\}]
{\color{outcolor}Out[{\color{outcolor}10}]:} 10
\end{Verbatim}
            

    % Add a bibliography block to the postdoc
    
    
    
    \end{document}
